%%
% Copyright (c) 2017 - 2025, Pascal Wagler;
% Copyright (c) 2014 - 2025, John MacFarlane
%
% All rights reserved.
%
% Redistribution and use in source and binary forms, with or without
% modification, are permitted provided that the following conditions
% are met:
%
% - Redistributions of source code must retain the above copyright
% notice, this list of conditions and the following disclaimer.
%
% - Redistributions in binary form must reproduce the above copyright
% notice, this list of conditions and the following disclaimer in the
% documentation and/or other materials provided with the distribution.
%
% - Neither the name of John MacFarlane nor the names of other
% contributors may be used to endorse or promote products derived
% from this software without specific prior written permission.
%
% THIS SOFTWARE IS PROVIDED BY THE COPYRIGHT HOLDERS AND CONTRIBUTORS
% "AS IS" AND ANY EXPRESS OR IMPLIED WARRANTIES, INCLUDING, BUT NOT
% LIMITED TO, THE IMPLIED WARRANTIES OF MERCHANTABILITY AND FITNESS
% FOR A PARTICULAR PURPOSE ARE DISCLAIMED. IN NO EVENT SHALL THE
% COPYRIGHT OWNER OR CONTRIBUTORS BE LIABLE FOR ANY DIRECT, INDIRECT,
% INCIDENTAL, SPECIAL, EXEMPLARY, OR CONSEQUENTIAL DAMAGES (INCLUDING,
% BUT NOT LIMITED TO, PROCUREMENT OF SUBSTITUTE GOODS OR SERVICES;
% LOSS OF USE, DATA, OR PROFITS; OR BUSINESS INTERRUPTION) HOWEVER
% CAUSED AND ON ANY THEORY OF LIABILITY, WHETHER IN CONTRACT, STRICT
% LIABILITY, OR TORT (INCLUDING NEGLIGENCE OR OTHERWISE) ARISING IN
% ANY WAY OUT OF THE USE OF THIS SOFTWARE, EVEN IF ADVISED OF THE
% POSSIBILITY OF SUCH DAMAGE.
%%

%%
% This is the Eisvogel pandoc LaTeX template.
%
% For usage information and examples visit the official GitHub page:
% https://github.com/Wandmalfarbe/pandoc-latex-template
%%
% Options for packages loaded elsewhere
\PassOptionsToPackage{unicode}{hyperref}
\PassOptionsToPackage{hyphens}{url}
\PassOptionsToPackage{dvipsnames,svgnames,x11names,table}{xcolor}
\documentclass[
  paper=a4,
  ,captions=tableheading
]{scrartcl}
\usepackage{xcolor}
\usepackage[margin=2.5cm,includehead=true,includefoot=true,centering,]{geometry}
\usepackage{amsmath,amssymb}

% add backlinks to footnote references, cf. https://tex.stackexchange.com/questions/302266/make-footnote-clickable-both-ways
\usepackage{footnotebackref}
\setcounter{secnumdepth}{5}
\usepackage{iftex}
\ifPDFTeX
  \usepackage[T1]{fontenc}
  \usepackage[utf8]{inputenc}
  \usepackage{textcomp} % provide euro and other symbols
\else % if luatex or xetex
  \usepackage{unicode-math} % this also loads fontspec
  \defaultfontfeatures{Scale=MatchLowercase}
  \defaultfontfeatures[\rmfamily]{Ligatures=TeX,Scale=1}
\fi
\usepackage{lmodern}
\ifPDFTeX\else
  % xetex/luatex font selection
\fi
% Use upquote if available, for straight quotes in verbatim environments
\IfFileExists{upquote.sty}{\usepackage{upquote}}{}
\IfFileExists{microtype.sty}{% use microtype if available
  \usepackage[]{microtype}
  \UseMicrotypeSet[protrusion]{basicmath} % disable protrusion for tt fonts
}{}

% Use setspace anyway because we change the default line spacing.
% The spacing is changed early to affect the titlepage and the TOC.
\usepackage{setspace}
\setstretch{1.2}
\makeatletter
\@ifundefined{KOMAClassName}{% if non-KOMA class
  \IfFileExists{parskip.sty}{%
    \usepackage{parskip}
  }{% else
    \setlength{\parindent}{0pt}
    \setlength{\parskip}{6pt plus 2pt minus 1pt}}
}{% if KOMA class
  \KOMAoptions{parskip=half}}
\makeatother
\usepackage{color}
\usepackage{fancyvrb}
\newcommand{\VerbBar}{|}
\newcommand{\VERB}{\Verb[commandchars=\\\{\}]}
\DefineVerbatimEnvironment{Highlighting}{Verbatim}{commandchars=\\\{\}}
% Add ',fontsize=\small' for more characters per line
\newenvironment{Shaded}{}{}
\newcommand{\AlertTok}[1]{\textcolor[rgb]{1.00,0.00,0.00}{\textbf{#1}}}
\newcommand{\AnnotationTok}[1]{\textcolor[rgb]{0.38,0.63,0.69}{\textbf{\textit{#1}}}}
\newcommand{\AttributeTok}[1]{\textcolor[rgb]{0.49,0.56,0.16}{#1}}
\newcommand{\BaseNTok}[1]{\textcolor[rgb]{0.25,0.63,0.44}{#1}}
\newcommand{\BuiltInTok}[1]{\textcolor[rgb]{0.00,0.50,0.00}{#1}}
\newcommand{\CharTok}[1]{\textcolor[rgb]{0.25,0.44,0.63}{#1}}
\newcommand{\CommentTok}[1]{\textcolor[rgb]{0.38,0.63,0.69}{\textit{#1}}}
\newcommand{\CommentVarTok}[1]{\textcolor[rgb]{0.38,0.63,0.69}{\textbf{\textit{#1}}}}
\newcommand{\ConstantTok}[1]{\textcolor[rgb]{0.53,0.00,0.00}{#1}}
\newcommand{\ControlFlowTok}[1]{\textcolor[rgb]{0.00,0.44,0.13}{\textbf{#1}}}
\newcommand{\DataTypeTok}[1]{\textcolor[rgb]{0.56,0.13,0.00}{#1}}
\newcommand{\DecValTok}[1]{\textcolor[rgb]{0.25,0.63,0.44}{#1}}
\newcommand{\DocumentationTok}[1]{\textcolor[rgb]{0.73,0.13,0.13}{\textit{#1}}}
\newcommand{\ErrorTok}[1]{\textcolor[rgb]{1.00,0.00,0.00}{\textbf{#1}}}
\newcommand{\ExtensionTok}[1]{#1}
\newcommand{\FloatTok}[1]{\textcolor[rgb]{0.25,0.63,0.44}{#1}}
\newcommand{\FunctionTok}[1]{\textcolor[rgb]{0.02,0.16,0.49}{#1}}
\newcommand{\ImportTok}[1]{\textcolor[rgb]{0.00,0.50,0.00}{\textbf{#1}}}
\newcommand{\InformationTok}[1]{\textcolor[rgb]{0.38,0.63,0.69}{\textbf{\textit{#1}}}}
\newcommand{\KeywordTok}[1]{\textcolor[rgb]{0.00,0.44,0.13}{\textbf{#1}}}
\newcommand{\NormalTok}[1]{#1}
\newcommand{\OperatorTok}[1]{\textcolor[rgb]{0.40,0.40,0.40}{#1}}
\newcommand{\OtherTok}[1]{\textcolor[rgb]{0.00,0.44,0.13}{#1}}
\newcommand{\PreprocessorTok}[1]{\textcolor[rgb]{0.74,0.48,0.00}{#1}}
\newcommand{\RegionMarkerTok}[1]{#1}
\newcommand{\SpecialCharTok}[1]{\textcolor[rgb]{0.25,0.44,0.63}{#1}}
\newcommand{\SpecialStringTok}[1]{\textcolor[rgb]{0.73,0.40,0.53}{#1}}
\newcommand{\StringTok}[1]{\textcolor[rgb]{0.25,0.44,0.63}{#1}}
\newcommand{\VariableTok}[1]{\textcolor[rgb]{0.10,0.09,0.49}{#1}}
\newcommand{\VerbatimStringTok}[1]{\textcolor[rgb]{0.25,0.44,0.63}{#1}}
\newcommand{\WarningTok}[1]{\textcolor[rgb]{0.38,0.63,0.69}{\textbf{\textit{#1}}}}

% Workaround/bugfix from jannick0.
% See https://github.com/jgm/pandoc/issues/4302#issuecomment-360669013)
% or https://github.com/Wandmalfarbe/pandoc-latex-template/issues/2
%
% Redefine the verbatim environment 'Highlighting' to break long lines (with
% the help of fvextra). Redefinition is necessary because it is unlikely that
% pandoc includes fvextra in the default template.
\usepackage{fvextra}
\DefineVerbatimEnvironment{Highlighting}{Verbatim}{breaklines,fontsize=\small,commandchars=\\\{\}}

\usepackage{longtable,booktabs,array}
\newcounter{none} % for unnumbered tables
\usepackage{calc} % for calculating minipage widths
% Correct order of tables after \paragraph or \subparagraph
\usepackage{etoolbox}
\makeatletter
\patchcmd\longtable{\par}{\if@noskipsec\mbox{}\fi\par}{}{}
\makeatother
% Allow footnotes in longtable head/foot
\IfFileExists{footnotehyper.sty}{\usepackage{footnotehyper}}{\usepackage{footnote}}
\makesavenoteenv{longtable}
\setlength{\emergencystretch}{3em} % prevent overfull lines
\providecommand{\tightlist}{%
  \setlength{\itemsep}{0pt}\setlength{\parskip}{0pt}}
\usepackage{bookmark}
\IfFileExists{xurl.sty}{\usepackage{xurl}}{} % add URL line breaks if available
\urlstyle{same}
\definecolor{default-linkcolor}{HTML}{A50000}
\definecolor{default-filecolor}{HTML}{A50000}
\definecolor{default-citecolor}{HTML}{4077C0}
\definecolor{default-urlcolor}{HTML}{4077C0}

\hypersetup{
  hidelinks,
  breaklinks=true,
  pdfcreator={LaTeX via pandoc with the Eisvogel template}}

\author{}
\date{}


%
% for the background color of the title page
%

%
% break urls
%
\PassOptionsToPackage{hyphens}{url}

%
% When using babel or polyglossia with biblatex, loading csquotes is recommended
% to ensure that quoted texts are typeset according to the rules of your main language.
%
\usepackage{csquotes}

%
% captions
%
\definecolor{caption-color}{HTML}{777777}
\usepackage[font={stretch=1.2}, textfont={color=caption-color}, position=top, skip=4mm, labelfont=bf, singlelinecheck=false, justification=raggedright]{caption}
\setcapindent{0em}

%
% blockquote
%
\definecolor{blockquote-border}{RGB}{221,221,221}
\definecolor{blockquote-text}{RGB}{119,119,119}
\usepackage{mdframed}
\newmdenv[rightline=false,bottomline=false,topline=false,linewidth=3pt,linecolor=blockquote-border,skipabove=\parskip]{customblockquote}
\renewenvironment{quote}{\begin{customblockquote}\list{}{\rightmargin=0em\leftmargin=0em}%
\item\relax\color{blockquote-text}\ignorespaces}{\unskip\unskip\endlist\end{customblockquote}}

%
% Source Sans Pro as the default font family
% Source Code Pro for monospace text
%
% 'default' option sets the default
% font family to Source Sans Pro, not \sfdefault.
%
% Note that the font has been officially renamed to `Source Sans 3`, and
% the version provided by the `sourcesanspro` package is slightly outdated.
% You can install the newer version locally and use it, for example, with
% `mainfont: "Source Sans 3"` in the YAML metadata (requires XeTeX or LuaTeX).
%
\ifnum 0\ifxetex 1\fi\ifluatex 1\fi=0 % if pdftex
    \usepackage[default]{sourcesanspro}
  \usepackage{sourcecodepro}
  \else % if not pdftex
    \usepackage[default]{sourcesanspro}
  \usepackage{sourcecodepro}

  % XeLaTeX specific adjustments for straight quotes: https://tex.stackexchange.com/a/354887
  % This issue is already fixed (see https://github.com/silkeh/latex-sourcecodepro/pull/5) but the
  % fix is still unreleased.
  % TODO: Remove this workaround when the new version of sourcecodepro is released on CTAN.
  \ifxetex
    \makeatletter
    \defaultfontfeatures[\ttfamily]
      { Numbers   = \sourcecodepro@figurestyle,
        Scale     = \SourceCodePro@scale,
        Extension = .otf }
    \setmonofont
      [ UprightFont    = *-\sourcecodepro@regstyle,
        ItalicFont     = *-\sourcecodepro@regstyle It,
        BoldFont       = *-\sourcecodepro@boldstyle,
        BoldItalicFont = *-\sourcecodepro@boldstyle It ]
      {SourceCodePro}
    \makeatother
  \fi
  \fi

%
% heading color
%
\definecolor{heading-color}{RGB}{40,40,40}
% By default, the KOMA-Script classes will typeset sectioning headings in
% sans-serif. Use the normal body font for headings.
\addtokomafont{disposition}{\normalfont\color{heading-color}\bfseries}

%
% variables for title, author and date
%
\usepackage{titling}
\title{}
\author{}
\date{}

%
% tables
%

\definecolor{table-row-color}{HTML}{F5F5F5}
\definecolor{table-rule-color}{HTML}{999999}

%\arrayrulecolor{black!40}
\arrayrulecolor{table-rule-color}     % color of \toprule, \midrule, \bottomrule
\setlength\heavyrulewidth{0.3ex}      % thickness of \toprule, \bottomrule
\renewcommand{\arraystretch}{1.3}     % spacing (padding)


%
% remove paragraph indentation
%
\setlength{\parindent}{0pt}
\setlength{\parskip}{6pt plus 2pt minus 1pt}
\setlength{\emergencystretch}{3em}  % prevent overfull lines

%
%
% Listings
%
%


%
% header and footer
%
\usepackage[headsepline,footsepline]{scrlayer-scrpage}

\newpairofpagestyles{eisvogel-header-footer}{
  \clearpairofpagestyles
  \ihead*{}
  \chead*{}
  \ohead*{}
  \ifoot*{}
  \cfoot*{}
  \ofoot*{\thepage}
  \addtokomafont{pageheadfoot}{\upshape}
}
\pagestyle{eisvogel-header-footer}



%
% Define watermark
%

\begin{document}




\hypertarget{wetware-engineering-applying-software-engineering-paradigms-to-biological-system-construction}{%
\section{Wetware Engineering: Applying Software Engineering Paradigms to
Biological System
Construction}\label{wetware-engineering-applying-software-engineering-paradigms-to-biological-system-construction}}

\textbf{A Cross-Disciplinary Methodology for Modular Life Systems}

\begin{center}\rule{0.5\linewidth}{0.5pt}\end{center}

\textbf{Author}: 123olp\\
\textbf{Email}: tukuai.ai@gmail.com\\
\textbf{ORCID}:
\href{https://orcid.org/0009-0009-6523-1823}{0009-0009-6523-1823}

\begin{center}\rule{0.5\linewidth}{0.5pt}\end{center}

\hypertarget{abstract}{%
\subsection{Abstract}\label{abstract}}

Software engineering underwent a paradigm shift from monolithic,
handcrafted programs to modular, composable systems over five
decades---a transformation enabled by standardized interfaces, package
managers, version control, and runtime orchestration. Biological
engineering, despite remarkable advances in synthetic biology,
organoids, and tissue engineering, remains trapped in an analogous
``pre-modular'' era: each biological system is constructed from scratch,
results are difficult to reproduce across laboratories, and there exists
no universal language for describing biological component composition.

We propose \textbf{Wetware Engineering}, a cross-disciplinary
methodology that systematically transfers software engineering's core
abstractions---modularity, interface standardization, dependency
management, and runtime orchestration---to biological system
construction. This is not merely applying computational tools to
biology, but fundamentally reconceptualizing how living systems should
be designed, described, and assembled.

Our contribution is threefold: (1) \textbf{Conceptual Framework}: We
define the Component-Interface-Runtime triad as the foundational
abstraction for modular biological systems, drawing explicit parallels
to software architecture patterns. (2) \textbf{Technical
Specifications}: We propose Bio-Component Spec, a standardized schema
for describing biological modules, and Bio-DSL, a domain-specific
language for declarative system composition---both designed following
software engineering best practices. (3) \textbf{Paradigm Analysis}: We
systematically analyze how software engineering concepts map to
biological contexts, identifying both direct translations and
fundamental differences requiring novel solutions.

Wetware Engineering represents a paradigm-level contribution: shifting
biological system construction from ``artisanal replication'' to
``engineered composition.'' While implementation faces significant
biological challenges, establishing this conceptual and methodological
foundation is a necessary first step toward reproducible, iterable, and
collaborative biological system development.

\textbf{Keywords}: Software Engineering, Biological Systems,
Cross-Disciplinary Methodology, Modular Design, Domain-Specific
Language, Systems Biology, Paradigm Transfer \# 1. Introduction: The
Case for Paradigm Transfer

\hypertarget{software-engineerings-modular-revolution}{%
\subsection{1.1 Software Engineering's Modular
Revolution}\label{software-engineerings-modular-revolution}}

The history of software engineering is fundamentally a history of rising
abstraction levels. In the 1950s, programmers wrote machine
code---sequences of binary instructions tied to specific hardware. The
introduction of assembly language provided the first abstraction:
human-readable mnemonics replacing numeric opcodes. Structured
programming in the 1960s abstracted control flow. Object-oriented
programming in the 1980s encapsulated data and behavior together.
Component-based development in the 1990s enabled binary-level reuse.
Service-oriented architecture in the 2000s abstracted deployment
locations. Microservices in the 2010s achieved independent deployment
and elastic scaling.

Each abstraction level brought transformative benefits:

\begin{longtable}[]{@{}
  >{\raggedright\arraybackslash}p{(\columnwidth - 6\tabcolsep) * \real{0.1190}}
  >{\raggedright\arraybackslash}p{(\columnwidth - 6\tabcolsep) * \real{0.3095}}
  >{\raggedright\arraybackslash}p{(\columnwidth - 6\tabcolsep) * \real{0.3810}}
  >{\raggedright\arraybackslash}p{(\columnwidth - 6\tabcolsep) * \real{0.1905}}@{}}
\toprule\noalign{}
\begin{minipage}[b]{\linewidth}\raggedright
Era
\end{minipage} & \begin{minipage}[b]{\linewidth}\raggedright
Abstraction
\end{minipage} & \begin{minipage}[b]{\linewidth}\raggedright
Key Innovation
\end{minipage} & \begin{minipage}[b]{\linewidth}\raggedright
Impact
\end{minipage} \\
\midrule\noalign{}
\endhead
\bottomrule\noalign{}
\endlastfoot
1950s & Machine code → Assembly & Human-readable instructions & 10x
productivity \\
1960s & Procedures & Structured programming & Manageable complexity \\
1970s & Modules & Information hiding, interfaces & Team collaboration \\
1980s & Objects & Data + behavior encapsulation & Reusable libraries \\
1990s & Components & Binary reuse (COM, JavaBeans) & Third-party
ecosystems \\
2000s & Services & Network-based composition & Enterprise integration \\
2010s & Microservices & Independent deployment & Cloud-native
scalability \\
\end{longtable}

The critical insight is that each abstraction level did not merely add
convenience---it fundamentally changed what was possible. Before package
managers like npm and pip, sharing code meant copying files and manually
resolving dependencies. Before containerization, ``it works on my
machine'' was an unsolvable problem. Before version control,
collaboration meant emailing zip files.

Today, a software developer can declare \texttt{import\ tensorflow} and
instantly access millions of lines of tested, documented,
version-controlled code. This is not magic---it is the accumulated
result of decades of standardization, tooling, and community building.

\hypertarget{biological-engineerings-pre-modular-state}{%
\subsection{1.2 Biological Engineering's ``Pre-Modular''
State}\label{biological-engineerings-pre-modular-state}}

Biological engineering in 2025, despite extraordinary advances, remains
in a state analogous to software engineering circa 1970. Consider the
following comparison:

\begin{longtable}[]{@{}
  >{\raggedright\arraybackslash}p{(\columnwidth - 4\tabcolsep) * \real{0.4688}}
  >{\raggedright\arraybackslash}p{(\columnwidth - 4\tabcolsep) * \real{0.3906}}
  >{\raggedright\arraybackslash}p{(\columnwidth - 4\tabcolsep) * \real{0.1406}}@{}}
\toprule\noalign{}
\begin{minipage}[b]{\linewidth}\raggedright
Software Engineering Concept
\end{minipage} & \begin{minipage}[b]{\linewidth}\raggedright
Current State in Biology
\end{minipage} & \begin{minipage}[b]{\linewidth}\raggedright
The Gap
\end{minipage} \\
\midrule\noalign{}
\endhead
\bottomrule\noalign{}
\endlastfoot
Standard Library & None & Each lab builds from scratch \\
Package Manager (npm, pip) & None & Cannot declare dependencies \\
Version Control (git) & None & ``This batch differs from last batch'' \\
API Documentation & None & ``Ask the original author how to culture
it'' \\
Unit Testing & None & ``How long will it last? Maybe a week'' \\
CI/CD Pipeline & None & No automated validation \\
Containerization (Docker) & None & Environments not reproducible \\
\end{longtable}

When a tissue engineer wants to combine a muscle actuator with a neural
controller, they face challenges that software engineers solved decades
ago:

\begin{enumerate}
\def\labelenumi{\arabic{enumi}.}
\item
  \textbf{No standard interfaces}: The muscle was developed in Lab A
  with specific culture conditions; the neural tissue in Lab B with
  different protocols. There is no guarantee they can physically or
  biochemically connect.
\item
  \textbf{No dependency declaration}: What exactly does the muscle need?
  Glucose concentration? Oxygen levels? Stimulation frequency? This
  information exists in lab notebooks, not machine-readable
  specifications.
\item
  \textbf{No version compatibility}: Lab A improved their muscle
  protocol last month. Does it still work with Lab B's neural tissue? No
  one knows without re-running experiments.
\item
  \textbf{No composition language}: How do you describe ``connect muscle
  output to sensor input, with closed-loop feedback control''? In
  natural language, buried in a methods section.
\end{enumerate}

The fundamental problem is conceptual: biological systems are treated as
\textbf{indivisible wholes} rather than \textbf{composable collections
of modules}.

\hypertarget{why-paradigm-transfer-not-just-tool-application}{%
\subsection{1.3 Why Paradigm Transfer, Not Just Tool
Application}\label{why-paradigm-transfer-not-just-tool-application}}

Existing ``computational biology'' primarily means: - Using computers to
\textbf{analyze} biological data (bioinformatics) - Using algorithms to
\textbf{simulate} biological processes (systems biology) - Using
software to \textbf{control} biological experiments (lab automation)

These are valuable but insufficient. They apply software as a tool to
biology, without changing how biology itself is engineered.

We propose something fundamentally different:

\begin{quote}
\textbf{Using software engineering's design philosophy to
reconceptualize how biological systems are constructed.}
\end{quote}

\begin{longtable}[]{@{}lll@{}}
\toprule\noalign{}
Level & Existing Approaches & Wetware Engineering \\
\midrule\noalign{}
\endhead
\bottomrule\noalign{}
\endlastfoot
Tool & Software analyzes biology & --- \\
Method & Algorithms optimize experiments & --- \\
\textbf{Paradigm} & --- & \textbf{Software thinking restructures
bioengineering} \\
\end{longtable}

The distinction matters. Tools and methods operate within existing
paradigms. Paradigm transfer creates new possibilities that were
previously inconceivable.

Consider an analogy: before the germ theory of disease, medicine
optimized treatments within a paradigm of ``balancing humors.'' Germ
theory did not merely add new treatments---it reconceptualized what
disease \emph{is}, enabling entirely new categories of intervention
(antibiotics, vaccines, sterilization).

Similarly, Wetware Engineering does not merely add new tools to
biological engineering. It reconceptualizes what a biological system
\emph{is}: not an indivisible organism, but a composable assembly of
standardized modules.

\hypertarget{contributions-and-paper-structure}{%
\subsection{1.4 Contributions and Paper
Structure}\label{contributions-and-paper-structure}}

This paper makes the following contributions:

\begin{enumerate}
\def\labelenumi{\arabic{enumi}.}
\item
  \textbf{Paradigm Definition}: We systematically propose transferring
  software engineering's core paradigms to biological system
  construction, articulating why this transfer is both necessary and
  feasible.
\item
  \textbf{Abstraction Framework}: We define the
  Component-Interface-Runtime triad as the foundational abstraction for
  modular biological systems, with explicit mappings to software
  architecture patterns.
\item
  \textbf{Technical Specifications}: We propose Bio-Component Spec v0.1,
  a standardized schema for describing biological modules, and Bio-DSL,
  a domain-specific language for declarative system composition.
\item
  \textbf{Mapping Analysis}: We systematically analyze how software
  engineering concepts translate to biological contexts, categorizing
  mappings as Direct, Analogous, or Novel (requiring new solutions).
\item
  \textbf{Difference Identification}: We identify fundamental
  differences between software and biological systems that require
  innovative approaches beyond direct paradigm transfer.
\end{enumerate}

The paper is structured as follows: - §2 defines core abstractions and
the Component-Interface-Runtime triad - §3 presents systematic mappings
from software to biological engineering - §4 details the Bio-Component
Specification design - §5 describes Bio-DSL language design rationale -
§6 analyzes fundamental differences and open challenges - §7 positions
our work relative to existing approaches - §8 concludes with future
directions \# 2. Core Abstractions: The Component-Interface-Runtime
Triad

\hypertarget{abstraction-as-the-essence-of-engineering}{%
\subsection{2.1 Abstraction as the Essence of
Engineering}\label{abstraction-as-the-essence-of-engineering}}

Edsger Dijkstra observed: ``The purpose of abstraction is not to be
vague, but to create a new semantic level in which one can be absolutely
precise.'' This insight captures why abstraction is not merely a
convenience but the essence of engineering progress.

Software engineering's success stems from identifying \textbf{correct
abstraction boundaries}: - Functions abstract instruction sequences -
Objects abstract data and behavior - Interfaces abstract implementation
details - Services abstract deployment locations - Containers abstract
operating environments

Each abstraction creates a ``semantic level'' where engineers can reason
precisely without concerning themselves with lower-level details. A web
developer using React does not think about memory allocation; a data
scientist using pandas does not think about CPU cache optimization.

The central question for biological engineering is: \textbf{What are the
correct abstraction boundaries for living systems?}

We propose that the answer lies in the same triad that revolutionized
software: \textbf{Component}, \textbf{Interface}, and \textbf{Runtime}.

\hypertarget{component-the-unit-of-biological-reuse}{%
\subsection{2.2 Component: The Unit of Biological
Reuse}\label{component-the-unit-of-biological-reuse}}

\hypertarget{definition}{%
\subsubsection{Definition}\label{definition}}

A \textbf{Bio-Component} is a biological unit that: - Can exist
independently (with appropriate life support) - Can receive energy and
nutrients (powerable) - Can respond to control signals (controllable) -
Can produce functional outputs (functional) - Can report its state
(observable)

This definition deliberately parallels software component definitions. A
software component is similarly self-contained, has defined inputs and
outputs, maintains internal state, and can be monitored.

\hypertarget{mapping-to-software-concepts}{%
\subsubsection{Mapping to Software
Concepts}\label{mapping-to-software-concepts}}

\begin{longtable}[]{@{}ll@{}}
\toprule\noalign{}
Software Component Property & Bio-Component Equivalent \\
\midrule\noalign{}
\endhead
\bottomrule\noalign{}
\endlastfoot
Encapsulation & Physical boundary, membrane structure \\
Interface & Input/output port definitions \\
State & Physiological state, viability indicators \\
Lifecycle & Culture, activation, maintenance, senescence \\
Dependencies & Nutrients, oxygen, signal inputs \\
Side Effects & Metabolic waste, secretions \\
\end{longtable}

\hypertarget{component-typology}{%
\subsubsection{Component Typology}\label{component-typology}}

Drawing from software architecture patterns, we propose a typology of
Bio-Components:

\begin{longtable}[]{@{}
  >{\raggedright\arraybackslash}p{(\columnwidth - 4\tabcolsep) * \real{0.1364}}
  >{\raggedright\arraybackslash}p{(\columnwidth - 4\tabcolsep) * \real{0.3864}}
  >{\raggedright\arraybackslash}p{(\columnwidth - 4\tabcolsep) * \real{0.4773}}@{}}
\toprule\noalign{}
\begin{minipage}[b]{\linewidth}\raggedright
Type
\end{minipage} & \begin{minipage}[b]{\linewidth}\raggedright
Software Analogy
\end{minipage} & \begin{minipage}[b]{\linewidth}\raggedright
Biological Examples
\end{minipage} \\
\midrule\noalign{}
\endhead
\bottomrule\noalign{}
\endlastfoot
\textbf{Actuator} & Output device driver & Muscle, gland, ciliated
epithelium \\
\textbf{Sensor} & Input device driver & Photoreceptor, mechanoreceptor,
chemoreceptor \\
\textbf{Processor} & CPU, logic unit & Ganglion, brain organoid, neural
network \\
\textbf{Storage} & Memory, database & Adipose tissue, bone marrow \\
\textbf{Connector} & Network interface & Blood vessel, nerve fiber \\
\textbf{Metabolic} & Power supply & Liver tissue, mitochondria-rich
cells \\
\end{longtable}

This typology is not exhaustive but illustrative. The key insight is
that biological structures can be categorized by their
\textbf{functional role} in a system, just as software components are
categorized by their architectural role.

\hypertarget{interface-the-contract-for-composition}{%
\subsection{2.3 Interface: The Contract for
Composition}\label{interface-the-contract-for-composition}}

\hypertarget{the-philosophy-of-interfaces}{%
\subsubsection{The Philosophy of
Interfaces}\label{the-philosophy-of-interfaces}}

The Gang of Four's design principle states: ``Program to an interface,
not an implementation.'' This principle enabled the explosion of
software reuse: as long as components agree on interfaces, their
internal implementations can vary independently.

An interface is a \textbf{contract} that defines: - What inputs are
accepted (preconditions) - What outputs are produced (postconditions) -
What guarantees are maintained (invariants)

The power of interfaces lies in \textbf{decoupling}: components can be
developed, tested, and replaced independently as long as they honor the
interface contract.

\hypertarget{bio-interface-dimensions}{%
\subsubsection{Bio-Interface
Dimensions}\label{bio-interface-dimensions}}

Biological interfaces are more complex than software interfaces because
they operate across multiple physical dimensions simultaneously. We
identify four primary dimensions:

\begin{verbatim}
┌─────────────────────────────────────────────────┐
│                 Bio-Interface                    │
├─────────────┬─────────────┬─────────────────────┤
│   Power     │   Signal    │    Isolation        │
│ (Energy)    │ (Information)│   (Barrier)        │
├─────────────┴─────────────┴─────────────────────┤
│            Mechanical (Force Coupling)           │
└─────────────────────────────────────────────────┘
\end{verbatim}

\textbf{Power Interface}: How energy and nutrients flow between
components - Perfusion connections (blood vessel equivalents) - Nutrient
diffusion surfaces - Oxygen delivery mechanisms

\textbf{Signal Interface}: How information is exchanged - Electrical
signals (neural) - Chemical signals (hormones, neurotransmitters) -
Mechanical signals (stretch, pressure) - Optical signals (for
optogenetic systems)

\textbf{Isolation Interface}: How components are protected from each
other - Immune barriers (preventing rejection) - Toxicity isolation
(containing harmful metabolites) - Electrical isolation (preventing
signal crosstalk)

\textbf{Mechanical Interface}: How physical forces are transmitted -
Structural attachments - Force transmission surfaces - Movement coupling

\hypertarget{the-usb-analogy}{%
\subsubsection{The USB Analogy}\label{the-usb-analogy}}

The success of USB (Universal Serial Bus) illustrates the transformative
power of interface standardization:

\begin{longtable}[]{@{}ll@{}}
\toprule\noalign{}
Before USB & After USB \\
\midrule\noalign{}
\endhead
\bottomrule\noalign{}
\endlastfoot
Every device had proprietary connectors & One connector fits all \\
Drivers required for each device & Plug-and-play \\
Limited ecosystem & Massive accessory market \\
Vendor lock-in & Consumer choice \\
\end{longtable}

We envision Bio-Interfaces achieving similar transformation: components
from different laboratories, different species, even synthetic origins,
connecting through standardized interfaces.

\hypertarget{runtime-the-orchestration-layer}{%
\subsection{2.4 Runtime: The Orchestration
Layer}\label{runtime-the-orchestration-layer}}

\hypertarget{software-runtime-responsibilities}{%
\subsubsection{Software Runtime
Responsibilities}\label{software-runtime-responsibilities}}

In software systems, the runtime environment handles: - \textbf{Resource
Management}: Memory allocation, CPU scheduling, network I/O -
\textbf{Lifecycle Management}: Starting, stopping, restarting components
- \textbf{Fault Handling}: Exception catching, recovery, graceful
degradation - \textbf{Monitoring}: Logging, metrics, health checks -
\textbf{Coordination}: Synchronization, communication routing

Modern container orchestrators like Kubernetes exemplify sophisticated
runtime systems, managing thousands of components across distributed
infrastructure.

\hypertarget{bio-runtime-responsibilities}{%
\subsubsection{Bio-Runtime
Responsibilities}\label{bio-runtime-responsibilities}}

A Bio-Runtime must handle analogous responsibilities in the biological
domain:

\begin{longtable}[]{@{}ll@{}}
\toprule\noalign{}
Software Runtime & Bio-Runtime \\
\midrule\noalign{}
\endhead
\bottomrule\noalign{}
\endlastfoot
Memory allocation & Nutrient allocation \\
CPU scheduling & Activity timing control \\
Network I/O & Signal routing \\
Health checks & Viability monitoring \\
Auto-restart & Regeneration/replacement triggering \\
Logging & Biomarker time-series recording \\
Load balancing & Workload distribution across redundant modules \\
Fault isolation & Containing necrosis, inflammation \\
\end{longtable}

\hypertarget{the-perfusion-system-as-infrastructure}{%
\subsubsection{The Perfusion System as
Infrastructure}\label{the-perfusion-system-as-infrastructure}}

Just as software components rely on operating system services (file
systems, network stacks), Bio-Components rely on life support
infrastructure. The perfusion system---delivering nutrients and oxygen
while removing waste---is the biological equivalent of power and network
infrastructure.

A Bio-Runtime must manage: - Flow rates and pressures - Temperature
regulation - pH maintenance - Oxygen levels - Waste removal - Growth
factor delivery

This is not merely ``keeping cells alive'' but actively orchestrating
the conditions under which components can function and interact.

\hypertarget{the-triad-in-action-a-conceptual-example}{%
\subsection{2.5 The Triad in Action: A Conceptual
Example}\label{the-triad-in-action-a-conceptual-example}}

Consider assembling a simple bio-robotic system: a muscle that contracts
in response to detected force.

\textbf{Components}: - Muscle actuator (Actuator type) - Force sensor
(Sensor type) - Neural controller (Processor type)

\textbf{Interfaces}: - Sensor → Controller: electrical signal interface
- Controller → Muscle: electrical stimulation interface - All
components: perfusion interface for nutrients

\textbf{Runtime}: - Perfusion system maintaining 37°C, pH 7.4 -
Monitoring system tracking viability and performance - Control loop
executing feedback algorithm

In software terms, this is analogous to:

\begin{verbatim}
sensor.onForceDetected(force => {
  controller.process(force);
  muscle.contract(controller.output);
});
\end{verbatim}

The Bio-DSL equivalent (detailed in §5):

\begin{verbatim}
CONNECT sensor.output TO controller.input
CONNECT controller.output TO muscle.stimulation
RUNTIME { perfusion: standard_mammalian, control: closed_loop }
\end{verbatim}

The power of this abstraction is that the same description could work
with: - Different muscle sources (human, mouse, synthetic) - Different
sensor technologies (piezoelectric, biological) - Different controller
implementations (organoid, silicon chip)

As long as interfaces are honored, components are interchangeable. \# 3.
Systematic Mapping: Software Engineering → Biological Engineering

\hypertarget{mapping-framework}{%
\subsection{3.1 Mapping Framework}\label{mapping-framework}}

Not all software engineering concepts transfer equally to biology. We
propose a three-category framework for analyzing mappings:

\begin{longtable}[]{@{}
  >{\raggedright\arraybackslash}p{(\columnwidth - 4\tabcolsep) * \real{0.2979}}
  >{\raggedright\arraybackslash}p{(\columnwidth - 4\tabcolsep) * \real{0.2553}}
  >{\raggedright\arraybackslash}p{(\columnwidth - 4\tabcolsep) * \real{0.4468}}@{}}
\toprule\noalign{}
\begin{minipage}[b]{\linewidth}\raggedright
Mapping Type
\end{minipage} & \begin{minipage}[b]{\linewidth}\raggedright
Definition
\end{minipage} & \begin{minipage}[b]{\linewidth}\raggedright
Transfer Difficulty
\end{minipage} \\
\midrule\noalign{}
\endhead
\bottomrule\noalign{}
\endlastfoot
\textbf{Direct} & Concept transfers with minimal adaptation & Low \\
\textbf{Analogous} & Core idea transfers but requires domain-specific
adaptation & Medium \\
\textbf{Novel} & No software equivalent; requires new solutions &
High \\
\end{longtable}

This categorization helps practitioners understand which software
practices can be immediately adopted, which require careful adaptation,
and which represent open research challenges.

\hypertarget{direct-mappings}{%
\subsection{3.2 Direct Mappings}\label{direct-mappings}}

These concepts can be transferred almost verbatim from software
engineering:

\hypertarget{semantic-versioning}{%
\subsubsection{Semantic Versioning}\label{semantic-versioning}}

Software's Semantic Versioning (SemVer) specification defines version
numbers as MAJOR.MINOR.PATCH: - MAJOR: incompatible API changes - MINOR:
backward-compatible functionality additions - PATCH: backward-compatible
bug fixes

This transfers directly to Bio-Components: - \textbf{MAJOR}:
Interface-incompatible changes (e.g., different input signal type) -
\textbf{MINOR}: Backward-compatible enhancements (e.g., improved force
output) - \textbf{PATCH}: Optimizations without interface changes (e.g.,
faster response time)

Example: \texttt{muscle-actuator-human-skeletal@2.3.1} - Version 2:
Second-generation interface (incompatible with v1) - .3: Third feature
addition since v2.0 - .1: First optimization patch

\hypertarget{dependency-declaration}{%
\subsubsection{Dependency Declaration}\label{dependency-declaration}}

Software package manifests (package.json, requirements.txt) declare
dependencies with version constraints:

\begin{Shaded}
\begin{Highlighting}[]
\FunctionTok{\{}
  \DataTypeTok{"dependencies"}\FunctionTok{:} \FunctionTok{\{}
    \DataTypeTok{"tensorflow"}\FunctionTok{:} \StringTok{"\^{}2.0.0"}\FunctionTok{,}
    \DataTypeTok{"numpy"}\FunctionTok{:} \StringTok{"\textgreater{}=1.19,\textless{}2.0"}
  \FunctionTok{\}}
\FunctionTok{\}}
\end{Highlighting}
\end{Shaded}

Bio-Component manifests can use identical syntax:

\begin{Shaded}
\begin{Highlighting}[]
\FunctionTok{dependencies}\KeywordTok{:}
\AttributeTok{  }\FunctionTok{perfusion{-}medium}\KeywordTok{:}\AttributeTok{ }\StringTok{"DMEM@\^{}1.0"}
\AttributeTok{  }\FunctionTok{oxygen{-}supply}\KeywordTok{:}\AttributeTok{ }\StringTok{"\textgreater{}=15\%"}
\AttributeTok{  }\FunctionTok{temperature{-}control}\KeywordTok{:}\AttributeTok{ }\StringTok{"37±2°C"}
\AttributeTok{  }\FunctionTok{co{-}culture}\KeywordTok{:}
\AttributeTok{    }\KeywordTok{{-}}\AttributeTok{ }\StringTok{"endothelial{-}cells@\^{}1.2"}\CommentTok{  \# for vascularization}
\end{Highlighting}
\end{Shaded}

\hypertarget{documentation-standards}{%
\subsubsection{Documentation Standards}\label{documentation-standards}}

README files, API documentation, and usage examples transfer directly. A
Bio-Component should include: - \textbf{Description}: What the component
does - \textbf{Requirements}: Environmental conditions needed -
\textbf{Interface Specification}: Inputs, outputs, parameters -
\textbf{Usage Examples}: How to integrate with other components -
\textbf{Known Limitations}: Failure modes, incompatibilities -
\textbf{Changelog}: Version history

\hypertarget{licensing}{%
\subsubsection{Licensing}\label{licensing}}

Software licenses (MIT, Apache, GPL) define usage rights. Biological
components need similar frameworks: - \textbf{Usage Rights}: Who can use
the component - \textbf{Modification Rights}: Can the component be
genetically modified? - \textbf{Sharing Requirements}: Must derivatives
be shared? - \textbf{Attribution}: How to credit original developers -
\textbf{Commercial Use}: Restrictions on commercial applications

\hypertarget{analogous-mappings}{%
\subsection{3.3 Analogous Mappings}\label{analogous-mappings}}

These concepts require adaptation but preserve core principles:

\hypertarget{testing-validation}{%
\subsubsection{Testing → Validation}\label{testing-validation}}

\begin{longtable}[]{@{}
  >{\raggedright\arraybackslash}p{(\columnwidth - 4\tabcolsep) * \real{0.2982}}
  >{\raggedright\arraybackslash}p{(\columnwidth - 4\tabcolsep) * \real{0.3860}}
  >{\raggedright\arraybackslash}p{(\columnwidth - 4\tabcolsep) * \real{0.3158}}@{}}
\toprule\noalign{}
\begin{minipage}[b]{\linewidth}\raggedright
Software Testing
\end{minipage} & \begin{minipage}[b]{\linewidth}\raggedright
Biological Validation
\end{minipage} & \begin{minipage}[b]{\linewidth}\raggedright
Adaptation Notes
\end{minipage} \\
\midrule\noalign{}
\endhead
\bottomrule\noalign{}
\endlastfoot
Unit Test & Viability Test & Test single component function \\
Integration Test & Compatibility Test & Test component interactions \\
Stress Test & Endurance Test & Long-term, extreme conditions \\
Regression Test & Batch Consistency Test & New batches match previous \\
Performance Test & Efficiency Test & Output per resource consumed \\
\end{longtable}

Key differences: - Software tests are deterministic; biological tests
are statistical - Software tests run in milliseconds; biological tests
take days/weeks - Software tests are automated; biological tests require
manual intervention

Adaptation: Define \textbf{acceptance criteria} as statistical
thresholds rather than exact values:

\begin{Shaded}
\begin{Highlighting}[]
\FunctionTok{tests}\KeywordTok{:}
\AttributeTok{  }\FunctionTok{viability}\KeywordTok{:}
\AttributeTok{    }\FunctionTok{metric}\KeywordTok{:}\AttributeTok{ }\StringTok{"cell\_survival\_rate"}
\AttributeTok{    }\FunctionTok{threshold}\KeywordTok{:}\AttributeTok{ }\StringTok{"\textgreater{}= 90\%"}
\AttributeTok{    }\FunctionTok{confidence}\KeywordTok{:}\AttributeTok{ }\StringTok{"95\%"}
\AttributeTok{    }\FunctionTok{sample\_size}\KeywordTok{:}\AttributeTok{ }\DecValTok{10}
\end{Highlighting}
\end{Shaded}

\hypertarget{api-contract-interface-protocol}{%
\subsubsection{API Contract → Interface
Protocol}\label{api-contract-interface-protocol}}

Software APIs define: - Input types and validation rules - Output types
and guarantees - Error conditions and handling

Bio-Interfaces define: - Input signal types and acceptable ranges -
Output characteristics and tolerances - Failure modes and detection
methods

Example software API:

\begin{Shaded}
\begin{Highlighting}[]
\KeywordTok{function} \FunctionTok{processSignal}\NormalTok{(input}\OperatorTok{:} \DataTypeTok{number}\NormalTok{)}\OperatorTok{:} \DataTypeTok{number}\NormalTok{ \{}
  \CommentTok{// Precondition: 0 \textless{}= input \textless{}= 100}
  \CommentTok{// Postcondition: returns processed value in [0, 1]}
  \CommentTok{// Throws: InvalidInputError if precondition violated}
\NormalTok{\}}
\end{Highlighting}
\end{Shaded}

Equivalent Bio-Interface:

\begin{Shaded}
\begin{Highlighting}[]
\FunctionTok{interface}\KeywordTok{:}
\AttributeTok{  }\FunctionTok{input}\KeywordTok{:}
\AttributeTok{    }\FunctionTok{type}\KeywordTok{:}\AttributeTok{ }\StringTok{"electrical"}
\AttributeTok{    }\FunctionTok{voltage}\KeywordTok{:}\AttributeTok{ }\KeywordTok{\{}\AttributeTok{ }\FunctionTok{min}\KeywordTok{:}\AttributeTok{ }\DecValTok{0}\KeywordTok{,}\AttributeTok{ }\FunctionTok{max}\KeywordTok{:}\AttributeTok{ }\DecValTok{5}\KeywordTok{,}\AttributeTok{ }\FunctionTok{unit}\KeywordTok{:}\AttributeTok{ }\StringTok{"V"}\AttributeTok{ }\KeywordTok{\}}
\AttributeTok{    }\FunctionTok{frequency}\KeywordTok{:}\AttributeTok{ }\KeywordTok{\{}\AttributeTok{ }\FunctionTok{min}\KeywordTok{:}\AttributeTok{ }\DecValTok{1}\KeywordTok{,}\AttributeTok{ }\FunctionTok{max}\KeywordTok{:}\AttributeTok{ }\DecValTok{100}\KeywordTok{,}\AttributeTok{ }\FunctionTok{unit}\KeywordTok{:}\AttributeTok{ }\StringTok{"Hz"}\AttributeTok{ }\KeywordTok{\}}
\AttributeTok{  }\FunctionTok{output}\KeywordTok{:}
\AttributeTok{    }\FunctionTok{type}\KeywordTok{:}\AttributeTok{ }\StringTok{"mechanical"}
\AttributeTok{    }\FunctionTok{force}\KeywordTok{:}\AttributeTok{ }\KeywordTok{\{}\AttributeTok{ }\FunctionTok{min}\KeywordTok{:}\AttributeTok{ }\DecValTok{0}\KeywordTok{,}\AttributeTok{ }\FunctionTok{max}\KeywordTok{:}\AttributeTok{ }\DecValTok{50}\KeywordTok{,}\AttributeTok{ }\FunctionTok{unit}\KeywordTok{:}\AttributeTok{ }\StringTok{"mN"}\AttributeTok{ }\KeywordTok{\}}
\AttributeTok{    }\FunctionTok{response\_time}\KeywordTok{:}\AttributeTok{ }\KeywordTok{\{}\AttributeTok{ }\FunctionTok{typical}\KeywordTok{:}\AttributeTok{ }\DecValTok{150}\KeywordTok{,}\AttributeTok{ }\FunctionTok{max}\KeywordTok{:}\AttributeTok{ }\DecValTok{300}\KeywordTok{,}\AttributeTok{ }\FunctionTok{unit}\KeywordTok{:}\AttributeTok{ }\StringTok{"ms"}\AttributeTok{ }\KeywordTok{\}}
\AttributeTok{  }\FunctionTok{failure\_modes}\KeywordTok{:}
\AttributeTok{    }\KeywordTok{{-}}\AttributeTok{ }\FunctionTok{condition}\KeywordTok{:}\AttributeTok{ }\StringTok{"voltage \textgreater{} 5V"}
\AttributeTok{      }\FunctionTok{result}\KeywordTok{:}\AttributeTok{ }\StringTok{"cell\_damage"}
\AttributeTok{      }\FunctionTok{detection}\KeywordTok{:}\AttributeTok{ }\StringTok{"impedance\_spike"}
\end{Highlighting}
\end{Shaded}

\hypertarget{error-handling-failure-mode-management}{%
\subsubsection{Error Handling → Failure Mode
Management}\label{error-handling-failure-mode-management}}

Software distinguishes: - \textbf{Exceptions}: Unexpected conditions
that can be caught and handled - \textbf{Errors}: Serious problems that
may require termination - \textbf{Warnings}: Non-critical issues that
should be logged

Biological systems have analogous categories: - \textbf{Recoverable
Degradation}: Temporary performance drop (fatigue) -
\textbf{Irreversible Damage}: Permanent function loss (necrosis) -
\textbf{Systemic Risk}: Threats to other components (inflammation,
infection)

\begin{Shaded}
\begin{Highlighting}[]
\FunctionTok{failure\_handling}\KeywordTok{:}
\AttributeTok{  }\FunctionTok{fatigue}\KeywordTok{:}
\AttributeTok{    }\FunctionTok{type}\KeywordTok{:}\AttributeTok{ }\StringTok{"recoverable"}
\AttributeTok{    }\FunctionTok{detection}\KeywordTok{:}\AttributeTok{ }\StringTok{"force\_output \textless{} 80\% baseline"}
\AttributeTok{    }\FunctionTok{response}\KeywordTok{:}\AttributeTok{ }\StringTok{"reduce\_stimulation\_frequency"}
\AttributeTok{    }\FunctionTok{recovery\_time}\KeywordTok{:}\AttributeTok{ }\StringTok{"2{-}4 hours"}
\AttributeTok{  }
\AttributeTok{  }\FunctionTok{necrosis}\KeywordTok{:}
\AttributeTok{    }\FunctionTok{type}\KeywordTok{:}\AttributeTok{ }\StringTok{"irreversible"}
\AttributeTok{    }\FunctionTok{detection}\KeywordTok{:}\AttributeTok{ }\StringTok{"viability \textless{} 50\%"}
\AttributeTok{    }\FunctionTok{response}\KeywordTok{:}\AttributeTok{ }\StringTok{"isolate\_and\_replace"}
\AttributeTok{    }\FunctionTok{propagation\_risk}\KeywordTok{:}\AttributeTok{ }\StringTok{"medium"}
\end{Highlighting}
\end{Shaded}

\hypertarget{logging-biomarker-recording}{%
\subsubsection{Logging → Biomarker
Recording}\label{logging-biomarker-recording}}

Software logging captures: - Timestamps - Event types (INFO, WARN,
ERROR) - Contextual data - Stack traces for debugging

Biological logging captures: - Timestamps - Physiological measurements -
Environmental conditions - Anomaly indicators

\begin{Shaded}
\begin{Highlighting}[]
\FunctionTok{logging}\KeywordTok{:}
\AttributeTok{  }\FunctionTok{continuous}\KeywordTok{:}
\AttributeTok{    }\KeywordTok{{-}}\AttributeTok{ }\FunctionTok{metric}\KeywordTok{:}\AttributeTok{ }\StringTok{"force\_output"}
\AttributeTok{      }\FunctionTok{frequency}\KeywordTok{:}\AttributeTok{ }\StringTok{"100 Hz"}
\AttributeTok{    }\KeywordTok{{-}}\AttributeTok{ }\FunctionTok{metric}\KeywordTok{:}\AttributeTok{ }\StringTok{"temperature"}
\AttributeTok{      }\FunctionTok{frequency}\KeywordTok{:}\AttributeTok{ }\StringTok{"1 Hz"}
\AttributeTok{  }
\AttributeTok{  }\FunctionTok{event\_triggered}\KeywordTok{:}
\AttributeTok{    }\KeywordTok{{-}}\AttributeTok{ }\FunctionTok{event}\KeywordTok{:}\AttributeTok{ }\StringTok{"stimulation"}
\AttributeTok{      }\FunctionTok{capture}\KeywordTok{:}\AttributeTok{ }\KeywordTok{[}\StringTok{"voltage"}\KeywordTok{,}\AttributeTok{ }\StringTok{"frequency"}\KeywordTok{,}\AttributeTok{ }\StringTok{"duration"}\KeywordTok{]}
\AttributeTok{    }\KeywordTok{{-}}\AttributeTok{ }\FunctionTok{event}\KeywordTok{:}\AttributeTok{ }\StringTok{"anomaly"}
\AttributeTok{      }\FunctionTok{capture}\KeywordTok{:}\AttributeTok{ }\KeywordTok{[}\StringTok{"all\_metrics"}\KeywordTok{,}\AttributeTok{ }\StringTok{"image\_snapshot"}\KeywordTok{]}
\end{Highlighting}
\end{Shaded}

\hypertarget{novel-challenges-requiring-new-solutions}{%
\subsection{3.4 Novel Challenges Requiring New
Solutions}\label{novel-challenges-requiring-new-solutions}}

These challenges have no direct software equivalent and require
innovative approaches:

\hypertarget{immune-compatibility}{%
\subsubsection{Immune Compatibility}\label{immune-compatibility}}

Software components do not ``reject'' each other. Biological components
from different sources may trigger immune responses.

\textbf{Required Innovation}: Immune Compatibility Protocol

\begin{Shaded}
\begin{Highlighting}[]
\FunctionTok{immune\_profile}\KeywordTok{:}
\AttributeTok{  }\FunctionTok{mhc\_class\_i}\KeywordTok{:}\AttributeTok{ }\KeywordTok{[}\StringTok{"HLA{-}A*02:01"}\KeywordTok{,}\AttributeTok{ }\StringTok{"HLA{-}B*07:02"}\KeywordTok{]}
\AttributeTok{  }\FunctionTok{mhc\_class\_ii}\KeywordTok{:}\AttributeTok{ }\KeywordTok{[}\StringTok{"HLA{-}DR*04:01"}\KeywordTok{]}
\AttributeTok{  }\FunctionTok{known\_antigens}\KeywordTok{:}\AttributeTok{ }\KeywordTok{[}\StringTok{"collagen\_type\_i"}\KeywordTok{]}
\AttributeTok{  }
\FunctionTok{compatibility\_requirements}\KeywordTok{:}
\AttributeTok{  }\FunctionTok{autologous}\KeywordTok{:}\AttributeTok{ }\StringTok{"preferred"}
\AttributeTok{  }\FunctionTok{allogeneic}\KeywordTok{:}\AttributeTok{ }\StringTok{"requires\_matching"}
\AttributeTok{  }\FunctionTok{xenogeneic}\KeywordTok{:}\AttributeTok{ }\StringTok{"requires\_isolation\_barrier"}
\end{Highlighting}
\end{Shaded}

\hypertarget{signal-crosstalk}{%
\subsubsection{Signal Crosstalk}\label{signal-crosstalk}}

Software processes are isolated by operating system memory protection.
Biological components share chemical environments where signals can
interfere.

\textbf{Required Innovation}: Biological Noise Isolation Specification

\begin{Shaded}
\begin{Highlighting}[]
\FunctionTok{signal\_isolation}\KeywordTok{:}
\AttributeTok{  }\FunctionTok{electrical}\KeywordTok{:}
\AttributeTok{    }\FunctionTok{shielding}\KeywordTok{:}\AttributeTok{ }\StringTok{"required"}
\AttributeTok{    }\FunctionTok{max\_crosstalk}\KeywordTok{:}\AttributeTok{ }\StringTok{"{-}40 dB"}
\AttributeTok{  }
\AttributeTok{  }\FunctionTok{chemical}\KeywordTok{:}
\AttributeTok{    }\FunctionTok{diffusion\_barrier}\KeywordTok{:}\AttributeTok{ }\StringTok{"hydrogel\_encapsulation"}
\AttributeTok{    }\FunctionTok{half\_life\_requirement}\KeywordTok{:}\AttributeTok{ }\StringTok{"\textless{} 10 seconds"}
\end{Highlighting}
\end{Shaded}

\hypertarget{metabolic-coupling}{%
\subsubsection{Metabolic Coupling}\label{metabolic-coupling}}

Software components consume CPU and memory independently. Biological
components share metabolic resources and produce waste that affects
neighbors.

\textbf{Required Innovation}: Metabolic Dependency Graph

\begin{Shaded}
\begin{Highlighting}[]
\FunctionTok{metabolism}\KeywordTok{:}
\AttributeTok{  }\FunctionTok{consumes}\KeywordTok{:}
\AttributeTok{    }\KeywordTok{{-}}\AttributeTok{ }\FunctionTok{glucose}\KeywordTok{:}\AttributeTok{ }\StringTok{"2.5 µmol/hour"}
\AttributeTok{    }\KeywordTok{{-}}\AttributeTok{ }\FunctionTok{oxygen}\KeywordTok{:}\AttributeTok{ }\StringTok{"5.0 µmol/hour"}
\AttributeTok{  }
\AttributeTok{  }\FunctionTok{produces}\KeywordTok{:}
\AttributeTok{    }\KeywordTok{{-}}\AttributeTok{ }\FunctionTok{lactate}\KeywordTok{:}\AttributeTok{ }\StringTok{"4.0 µmol/hour"}
\AttributeTok{    }\KeywordTok{{-}}\AttributeTok{ }\FunctionTok{co2}\KeywordTok{:}\AttributeTok{ }\StringTok{"4.5 µmol/hour"}
\AttributeTok{  }
\AttributeTok{  }\FunctionTok{toxic\_threshold}\KeywordTok{:}
\AttributeTok{    }\FunctionTok{lactate}\KeywordTok{:}\AttributeTok{ }\StringTok{"\textless{} 20 mM in shared medium"}
\end{Highlighting}
\end{Shaded}

\hypertarget{living-degradation}{%
\subsubsection{Living Degradation}\label{living-degradation}}

Software does not age (though it can become obsolete). Biological
components inherently degrade over time.

\textbf{Required Innovation}: Degradation Prediction Model

\begin{Shaded}
\begin{Highlighting}[]
\FunctionTok{degradation}\KeywordTok{:}
\AttributeTok{  }\FunctionTok{expected\_lifetime}\KeywordTok{:}
\AttributeTok{    }\FunctionTok{mean}\KeywordTok{:}\AttributeTok{ }\DecValTok{14}
\AttributeTok{    }\FunctionTok{std}\KeywordTok{:}\AttributeTok{ }\DecValTok{3}
\AttributeTok{    }\FunctionTok{unit}\KeywordTok{:}\AttributeTok{ }\StringTok{"days"}
\AttributeTok{  }
\AttributeTok{  }\FunctionTok{degradation\_markers}\KeywordTok{:}
\AttributeTok{    }\KeywordTok{{-}}\AttributeTok{ }\FunctionTok{marker}\KeywordTok{:}\AttributeTok{ }\StringTok{"force\_decline"}
\AttributeTok{      }\FunctionTok{threshold}\KeywordTok{:}\AttributeTok{ }\StringTok{"20\% from baseline"}
\AttributeTok{      }\FunctionTok{interpretation}\KeywordTok{:}\AttributeTok{ }\StringTok{"approaching\_end\_of\_life"}
\AttributeTok{  }
\AttributeTok{  }\FunctionTok{replacement\_protocol}\KeywordTok{:}
\AttributeTok{    }\FunctionTok{trigger}\KeywordTok{:}\AttributeTok{ }\StringTok{"viability \textless{} 70\%"}
\AttributeTok{    }\FunctionTok{procedure}\KeywordTok{:}\AttributeTok{ }\StringTok{"hot\_swap\_with\_backup"}
\end{Highlighting}
\end{Shaded}

\hypertarget{ethical-constraints}{%
\subsubsection{Ethical Constraints}\label{ethical-constraints}}

Software has no inherent ethical status. Biological components,
especially those derived from humans or involving neural tissue, raise
ethical considerations with no software parallel.

\textbf{Required Innovation}: Ethical Constraint Language

\begin{Shaded}
\begin{Highlighting}[]
\FunctionTok{ethics}\KeywordTok{:}
\AttributeTok{  }\FunctionTok{source}\KeywordTok{:}
\AttributeTok{    }\FunctionTok{consent}\KeywordTok{:}\AttributeTok{ }\StringTok{"informed\_consent\_documented"}
\AttributeTok{    }\FunctionTok{donor\_type}\KeywordTok{:}\AttributeTok{ }\StringTok{"adult\_volunteer"}
\AttributeTok{  }
\AttributeTok{  }\FunctionTok{constraints}\KeywordTok{:}
\AttributeTok{    }\KeywordTok{{-}}\AttributeTok{ }\StringTok{"no\_consciousness\_capable\_assemblies"}
\AttributeTok{    }\KeywordTok{{-}}\AttributeTok{ }\StringTok{"no\_reproductive\_applications"}
\AttributeTok{    }\KeywordTok{{-}}\AttributeTok{ }\StringTok{"destruction\_protocol\_required"}
\AttributeTok{  }
\AttributeTok{  }\FunctionTok{oversight}\KeywordTok{:}
\AttributeTok{    }\FunctionTok{irb\_approval}\KeywordTok{:}\AttributeTok{ }\StringTok{"required"}
\AttributeTok{    }\FunctionTok{review\_frequency}\KeywordTok{:}\AttributeTok{ }\StringTok{"annual"}
\end{Highlighting}
\end{Shaded}

\hypertarget{bio-component-specification-design-rationale}{%
\section{4. Bio-Component Specification: Design
Rationale}\label{bio-component-specification-design-rationale}}

\hypertarget{design-principles-from-software-engineering}{%
\subsection{4.1 Design Principles from Software
Engineering}\label{design-principles-from-software-engineering}}

The Bio-Component Specification draws from established software
engineering principles:

\hypertarget{solid-principles-applied}{%
\subsubsection{SOLID Principles
Applied}\label{solid-principles-applied}}

\begin{longtable}[]{@{}
  >{\raggedright\arraybackslash}p{(\columnwidth - 4\tabcolsep) * \real{0.1833}}
  >{\raggedright\arraybackslash}p{(\columnwidth - 4\tabcolsep) * \real{0.3333}}
  >{\raggedright\arraybackslash}p{(\columnwidth - 4\tabcolsep) * \real{0.4833}}@{}}
\toprule\noalign{}
\begin{minipage}[b]{\linewidth}\raggedright
Principle
\end{minipage} & \begin{minipage}[b]{\linewidth}\raggedright
Software Definition
\end{minipage} & \begin{minipage}[b]{\linewidth}\raggedright
Bio-Component Application
\end{minipage} \\
\midrule\noalign{}
\endhead
\bottomrule\noalign{}
\endlastfoot
\textbf{Single Responsibility} & A class should have one reason to
change & A component should perform one biological function \\
\textbf{Open/Closed} & Open for extension, closed for modification &
Components can be enhanced without changing interfaces \\
\textbf{Liskov Substitution} & Subtypes must be substitutable for base
types & Compatible components must be interchangeable \\
\textbf{Interface Segregation} & Many specific interfaces over one
general & Fine-grained interface definitions \\
\textbf{Dependency Inversion} & Depend on abstractions, not concretions
& Depend on interface specs, not specific implementations \\
\end{longtable}

\hypertarget{convention-over-configuration}{%
\subsubsection{Convention over
Configuration}\label{convention-over-configuration}}

Borrowed from Ruby on Rails: provide sensible defaults to minimize
required configuration.

\begin{Shaded}
\begin{Highlighting}[]
\CommentTok{\# Minimal specification (defaults applied)}
\FunctionTok{component}\KeywordTok{:}
\AttributeTok{  }\FunctionTok{id}\KeywordTok{:}\AttributeTok{ }\StringTok{"muscle{-}actuator{-}v1"}
\AttributeTok{  }\FunctionTok{type}\KeywordTok{:}\AttributeTok{ }\StringTok{"actuator"}
\AttributeTok{  }\FunctionTok{source}\KeywordTok{:}\AttributeTok{ }\StringTok{"human{-}skeletal"}

\CommentTok{\# System applies defaults:}
\CommentTok{\# {-} temperature: 37°C}
\CommentTok{\# {-} pH: 7.4}
\CommentTok{\# {-} oxygen: 20\%}
\CommentTok{\# {-} standard mammalian culture medium}
\end{Highlighting}
\end{Shaded}

\hypertarget{schema-first-design}{%
\subsubsection{Schema-First Design}\label{schema-first-design}}

Like OpenAPI/Swagger for REST APIs, we define the schema before
implementations:

\begin{Shaded}
\begin{Highlighting}[]
\CommentTok{\# Bio{-}Component Spec Schema (JSON Schema format)}
\FunctionTok{$schema}\KeywordTok{:}\AttributeTok{ }\StringTok{"https://wetware{-}engineering.org/schema/bio{-}component/1.0"}
\FunctionTok{type}\KeywordTok{:}\AttributeTok{ object}
\FunctionTok{required}\KeywordTok{:}\AttributeTok{ }\KeywordTok{[}\AttributeTok{id}\KeywordTok{,}\AttributeTok{ name}\KeywordTok{,}\AttributeTok{ version}\KeywordTok{,}\AttributeTok{ type}\KeywordTok{,}\AttributeTok{ interface}\KeywordTok{]}
\FunctionTok{properties}\KeywordTok{:}
\AttributeTok{  }\FunctionTok{id}\KeywordTok{:}
\AttributeTok{    }\FunctionTok{type}\KeywordTok{:}\AttributeTok{ string}
\AttributeTok{    }\FunctionTok{pattern}\KeywordTok{:}\AttributeTok{ }\StringTok{"\^{}[a{-}z0{-}9{-}]+$"}
\AttributeTok{  }\FunctionTok{version}\KeywordTok{:}
\AttributeTok{    }\FunctionTok{type}\KeywordTok{:}\AttributeTok{ string}
\AttributeTok{    }\FunctionTok{pattern}\KeywordTok{:}\AttributeTok{ }\StringTok{"\^{}}\SpecialCharTok{\textbackslash{}\textbackslash{}}\StringTok{d+}\SpecialCharTok{\textbackslash{}\textbackslash{}}\StringTok{.}\SpecialCharTok{\textbackslash{}\textbackslash{}}\StringTok{d+}\SpecialCharTok{\textbackslash{}\textbackslash{}}\StringTok{.}\SpecialCharTok{\textbackslash{}\textbackslash{}}\StringTok{d+$"}
\end{Highlighting}
\end{Shaded}

\hypertarget{specification-structure}{%
\subsection{4.2 Specification Structure}\label{specification-structure}}

\hypertarget{complete-schema-overview}{%
\subsubsection{Complete Schema
Overview}\label{complete-schema-overview}}

\begin{Shaded}
\begin{Highlighting}[]
\FunctionTok{bio{-}component}\KeywordTok{:}\AttributeTok{ }\StringTok{"1.0"}\CommentTok{  \# Spec version}

\CommentTok{\# === IDENTIFICATION ===}
\FunctionTok{info}\KeywordTok{:}
\AttributeTok{  }\FunctionTok{id}\KeywordTok{:}\AttributeTok{ string}\CommentTok{                    \# Unique identifier}
\AttributeTok{  }\FunctionTok{name}\KeywordTok{:}\AttributeTok{ string}\CommentTok{                  \# Human{-}readable name}
\AttributeTok{  }\FunctionTok{version}\KeywordTok{:}\AttributeTok{ string}\CommentTok{               \# Semantic version}
\AttributeTok{  }\FunctionTok{description}\KeywordTok{:}\AttributeTok{ string}\CommentTok{           \# Brief description}
\AttributeTok{  }\FunctionTok{license}\KeywordTok{:}\AttributeTok{ string}\CommentTok{               \# Usage license}
\AttributeTok{  }\FunctionTok{authors}\KeywordTok{:}\AttributeTok{ }\KeywordTok{[}\AttributeTok{string}\KeywordTok{]}\CommentTok{             \# Contributors}
\AttributeTok{  }\FunctionTok{repository}\KeywordTok{:}\AttributeTok{ url}\CommentTok{               \# Source repository}

\CommentTok{\# === CLASSIFICATION ===}
\FunctionTok{classification}\KeywordTok{:}
\AttributeTok{  }\FunctionTok{type}\KeywordTok{:}\AttributeTok{ enum [actuator, sensor, processor, metabolic, structural, connector]}
\AttributeTok{  }\FunctionTok{domain}\KeywordTok{:}\AttributeTok{ string}\CommentTok{                \# e.g., "musculoskeletal", "neural"}
\AttributeTok{  }\FunctionTok{tags}\KeywordTok{:}\AttributeTok{ }\KeywordTok{[}\AttributeTok{string}\KeywordTok{]}\CommentTok{                \# Searchable tags}

\CommentTok{\# === BIOLOGICAL SOURCE ===}
\FunctionTok{source}\KeywordTok{:}
\AttributeTok{  }\FunctionTok{organism}\KeywordTok{:}\AttributeTok{ string}\CommentTok{              \# e.g., "Homo sapiens"}
\AttributeTok{  }\FunctionTok{tissue\_type}\KeywordTok{:}\AttributeTok{ string}\CommentTok{           \# e.g., "skeletal muscle"}
\AttributeTok{  }\FunctionTok{cell\_types}\KeywordTok{:}\AttributeTok{ }\KeywordTok{[}\AttributeTok{string}\KeywordTok{]}\CommentTok{          \# e.g., ["myocyte", "fibroblast"]}
\AttributeTok{  }\FunctionTok{cell\_line}\KeywordTok{:}\AttributeTok{ string}\CommentTok{             \# If immortalized}
\AttributeTok{  }\FunctionTok{genetic\_modifications}\KeywordTok{:}\AttributeTok{ }\KeywordTok{[}\AttributeTok{string}\KeywordTok{]}
\AttributeTok{  }\FunctionTok{culture\_protocol}\KeywordTok{:}\AttributeTok{ url}\CommentTok{         \# Reference to protocol}
\AttributeTok{  }\FunctionTok{biosafety\_level}\KeywordTok{:}\AttributeTok{ enum [BSL{-}1, BSL{-}2, BSL{-}3]}

\CommentTok{\# === INTERFACE DEFINITION ===}
\FunctionTok{interface}\KeywordTok{:}
\AttributeTok{  }\FunctionTok{inputs}\KeywordTok{:}\AttributeTok{ }\KeywordTok{[}\AttributeTok{InputPort}\KeywordTok{]}
\AttributeTok{  }\FunctionTok{outputs}\KeywordTok{:}\AttributeTok{ }\KeywordTok{[}\AttributeTok{OutputPort}\KeywordTok{]}
\AttributeTok{  }
\CommentTok{\# === ENVIRONMENTAL REQUIREMENTS ===}
\FunctionTok{requirements}\KeywordTok{:}
\AttributeTok{  }\FunctionTok{physical}\KeywordTok{:}\AttributeTok{ PhysicalRequirements}
\AttributeTok{  }\FunctionTok{chemical}\KeywordTok{:}\AttributeTok{ ChemicalRequirements}
\AttributeTok{  }\FunctionTok{biological}\KeywordTok{:}\AttributeTok{ BiologicalRequirements}

\CommentTok{\# === PERFORMANCE CHARACTERISTICS ===}
\FunctionTok{performance}\KeywordTok{:}
\AttributeTok{  }\FunctionTok{functional}\KeywordTok{:}\AttributeTok{ FunctionalMetrics}
\AttributeTok{  }\FunctionTok{reliability}\KeywordTok{:}\AttributeTok{ ReliabilityMetrics}
\AttributeTok{  }\FunctionTok{resources}\KeywordTok{:}\AttributeTok{ ResourceConsumption}

\CommentTok{\# === FAILURE MODES ===}
\FunctionTok{failure\_modes}\KeywordTok{:}\AttributeTok{ }\KeywordTok{[}\AttributeTok{FailureMode}\KeywordTok{]}

\CommentTok{\# === }\AlertTok{TESTING}\CommentTok{ ===}
\FunctionTok{testing}\KeywordTok{:}
\AttributeTok{  }\FunctionTok{unit\_tests}\KeywordTok{:}\AttributeTok{ }\KeywordTok{[}\AttributeTok{TestCase}\KeywordTok{]}
\AttributeTok{  }\FunctionTok{integration\_tests}\KeywordTok{:}\AttributeTok{ }\KeywordTok{[}\AttributeTok{IntegrationTest}\KeywordTok{]}
\AttributeTok{  }\FunctionTok{certification}\KeywordTok{:}\AttributeTok{ CertificationStatus}

\CommentTok{\# === DEPENDENCIES ===}
\FunctionTok{dependencies}\KeywordTok{:}
\AttributeTok{  }\FunctionTok{components}\KeywordTok{:}\AttributeTok{ }\KeywordTok{[}\AttributeTok{ComponentDependency}\KeywordTok{]}
\AttributeTok{  }\FunctionTok{adapters}\KeywordTok{:}\AttributeTok{ }\KeywordTok{[}\AttributeTok{AdapterDependency}\KeywordTok{]}
\AttributeTok{  }\FunctionTok{protocols}\KeywordTok{:}\AttributeTok{ }\KeywordTok{[}\AttributeTok{ProtocolDependency}\KeywordTok{]}
\end{Highlighting}
\end{Shaded}

\hypertarget{inputoutput-port-definition}{%
\subsubsection{Input/Output Port
Definition}\label{inputoutput-port-definition}}

\begin{Shaded}
\begin{Highlighting}[]
\FunctionTok{InputPort}\KeywordTok{:}
\AttributeTok{  }\FunctionTok{id}\KeywordTok{:}\AttributeTok{ string}\CommentTok{                    \# Port identifier}
\AttributeTok{  }\FunctionTok{type}\KeywordTok{:}\AttributeTok{ enum [electrical, chemical, mechanical, optical, thermal, perfusion]}
\AttributeTok{  }\FunctionTok{required}\KeywordTok{:}\AttributeTok{ boolean}\CommentTok{             \# Is this input mandatory?}
\AttributeTok{  }
\AttributeTok{  }\FunctionTok{parameters}\KeywordTok{:}
\AttributeTok{    }\FunctionTok{range}\KeywordTok{:}\AttributeTok{ }\KeywordTok{[}\AttributeTok{min}\KeywordTok{,}\AttributeTok{ max}\KeywordTok{]}\CommentTok{           \# Acceptable input range}
\AttributeTok{    }\FunctionTok{unit}\KeywordTok{:}\AttributeTok{ string}\CommentTok{                \# Physical unit}
\AttributeTok{    }\FunctionTok{resolution}\KeywordTok{:}\AttributeTok{ number}\CommentTok{          \# Minimum detectable change}
\AttributeTok{    }\FunctionTok{response\_time}\KeywordTok{:}\AttributeTok{ number}\CommentTok{       \# Time to respond (ms)}
\AttributeTok{  }
\AttributeTok{  }\FunctionTok{defaults}\KeywordTok{:}
\AttributeTok{    }\FunctionTok{value}\KeywordTok{:}\AttributeTok{ any}\CommentTok{                  \# Default if not connected}
\AttributeTok{  }
\AttributeTok{  }\FunctionTok{validation}\KeywordTok{:}
\AttributeTok{    }\FunctionTok{constraints}\KeywordTok{:}\AttributeTok{ }\KeywordTok{[}\AttributeTok{string}\KeywordTok{]}\CommentTok{       \# e.g., "frequency \textless{} 100 Hz"}

\FunctionTok{OutputPort}\KeywordTok{:}
\AttributeTok{  }\FunctionTok{id}\KeywordTok{:}\AttributeTok{ string}
\AttributeTok{  }\FunctionTok{type}\KeywordTok{:}\AttributeTok{ enum [electrical, chemical, mechanical, optical, thermal, secretion]}
\AttributeTok{  }
\AttributeTok{  }\FunctionTok{parameters}\KeywordTok{:}
\AttributeTok{    }\FunctionTok{range}\KeywordTok{:}\AttributeTok{ }\KeywordTok{[}\AttributeTok{min}\KeywordTok{,}\AttributeTok{ max}\KeywordTok{]}
\AttributeTok{    }\FunctionTok{unit}\KeywordTok{:}\AttributeTok{ string}
\AttributeTok{    }\FunctionTok{precision}\KeywordTok{:}\AttributeTok{ number}\CommentTok{           \# Output precision}
\AttributeTok{    }\FunctionTok{latency}\KeywordTok{:}\AttributeTok{ number}\CommentTok{             \# Output delay (ms)}
\AttributeTok{  }
\AttributeTok{  }\FunctionTok{monitoring}\KeywordTok{:}
\AttributeTok{    }\FunctionTok{metrics}\KeywordTok{:}\AttributeTok{ }\KeywordTok{[}\AttributeTok{string}\KeywordTok{]}\CommentTok{           \# Observable metrics}
\AttributeTok{    }\FunctionTok{sampling\_rate}\KeywordTok{:}\AttributeTok{ number}\CommentTok{       \# Hz}
\end{Highlighting}
\end{Shaded}

\hypertarget{environmental-requirements}{%
\subsubsection{Environmental
Requirements}\label{environmental-requirements}}

\begin{Shaded}
\begin{Highlighting}[]
\FunctionTok{PhysicalRequirements}\KeywordTok{:}
\AttributeTok{  }\FunctionTok{temperature}\KeywordTok{:}
\AttributeTok{    }\FunctionTok{optimal}\KeywordTok{:}\AttributeTok{ number}
\AttributeTok{    }\FunctionTok{range}\KeywordTok{:}\AttributeTok{ }\KeywordTok{[}\AttributeTok{min}\KeywordTok{,}\AttributeTok{ max}\KeywordTok{]}
\AttributeTok{    }\FunctionTok{unit}\KeywordTok{:}\AttributeTok{ }\StringTok{"°C"}
\AttributeTok{    }\FunctionTok{tolerance}\KeywordTok{:}\AttributeTok{ number}\CommentTok{           \# Acceptable deviation}
\AttributeTok{  }
\AttributeTok{  }\FunctionTok{pressure}\KeywordTok{:}
\AttributeTok{    }\FunctionTok{range}\KeywordTok{:}\AttributeTok{ }\KeywordTok{[}\AttributeTok{min}\KeywordTok{,}\AttributeTok{ max}\KeywordTok{]}
\AttributeTok{    }\FunctionTok{unit}\KeywordTok{:}\AttributeTok{ }\StringTok{"mmHg"}
\AttributeTok{  }
\AttributeTok{  }\FunctionTok{humidity}\KeywordTok{:}
\AttributeTok{    }\FunctionTok{range}\KeywordTok{:}\AttributeTok{ }\KeywordTok{[}\AttributeTok{min}\KeywordTok{,}\AttributeTok{ max}\KeywordTok{]}
\AttributeTok{    }\FunctionTok{unit}\KeywordTok{:}\AttributeTok{ }\StringTok{"\%"}

\FunctionTok{ChemicalRequirements}\KeywordTok{:}
\AttributeTok{  }\FunctionTok{pH}\KeywordTok{:}
\AttributeTok{    }\FunctionTok{optimal}\KeywordTok{:}\AttributeTok{ number}
\AttributeTok{    }\FunctionTok{range}\KeywordTok{:}\AttributeTok{ }\KeywordTok{[}\AttributeTok{min}\KeywordTok{,}\AttributeTok{ max}\KeywordTok{]}
\AttributeTok{  }
\AttributeTok{  }\FunctionTok{osmolarity}\KeywordTok{:}
\AttributeTok{    }\FunctionTok{range}\KeywordTok{:}\AttributeTok{ }\KeywordTok{[}\AttributeTok{min}\KeywordTok{,}\AttributeTok{ max}\KeywordTok{]}
\AttributeTok{    }\FunctionTok{unit}\KeywordTok{:}\AttributeTok{ }\StringTok{"mOsm/L"}
\AttributeTok{  }
\AttributeTok{  }\FunctionTok{oxygen}\KeywordTok{:}
\AttributeTok{    }\FunctionTok{range}\KeywordTok{:}\AttributeTok{ }\KeywordTok{[}\AttributeTok{min}\KeywordTok{,}\AttributeTok{ max}\KeywordTok{]}
\AttributeTok{    }\FunctionTok{unit}\KeywordTok{:}\AttributeTok{ }\StringTok{"\%"}
\AttributeTok{  }
\AttributeTok{  }\FunctionTok{glucose}\KeywordTok{:}
\AttributeTok{    }\FunctionTok{range}\KeywordTok{:}\AttributeTok{ }\KeywordTok{[}\AttributeTok{min}\KeywordTok{,}\AttributeTok{ max}\KeywordTok{]}
\AttributeTok{    }\FunctionTok{unit}\KeywordTok{:}\AttributeTok{ }\StringTok{"mM"}
\AttributeTok{  }
\AttributeTok{  }\FunctionTok{custom\_factors}\KeywordTok{:}
\AttributeTok{    }\KeywordTok{{-}}\AttributeTok{ }\FunctionTok{name}\KeywordTok{:}\AttributeTok{ string}
\AttributeTok{      }\FunctionTok{concentration}\KeywordTok{:}\AttributeTok{ number}
\AttributeTok{      }\FunctionTok{unit}\KeywordTok{:}\AttributeTok{ string}

\FunctionTok{BiologicalRequirements}\KeywordTok{:}
\AttributeTok{  }\FunctionTok{immune\_isolation}\KeywordTok{:}\AttributeTok{ boolean}
\AttributeTok{  }\FunctionTok{sterility}\KeywordTok{:}\AttributeTok{ enum [sterile, clean, standard]}
\AttributeTok{  }\FunctionTok{co\_culture\_requirements}\KeywordTok{:}\AttributeTok{ }\KeywordTok{[}\AttributeTok{string}\KeywordTok{]}
\end{Highlighting}
\end{Shaded}

\hypertarget{versioning-strategy}{%
\subsection{4.3 Versioning Strategy}\label{versioning-strategy}}

\hypertarget{semantic-versioning-for-biology}{%
\subsubsection{Semantic Versioning for
Biology}\label{semantic-versioning-for-biology}}

We adopt SemVer with biological interpretations:

\textbf{MAJOR version} (X.0.0): Interface-breaking changes -
Input/output port type changes - Required parameter additions -
Environmental requirement changes that affect compatibility

\textbf{MINOR version} (0.X.0): Backward-compatible additions - New
optional output ports - Performance improvements - Additional monitoring
capabilities

\textbf{PATCH version} (0.0.X): Backward-compatible fixes - Protocol
optimizations - Documentation updates - Minor performance tuning

\hypertarget{build-metadata}{%
\subsubsection{Build Metadata}\label{build-metadata}}

Extended version format for biological specificity:

\begin{verbatim}
{version}+{batch}.{donor}.{modification}

Examples:
2.3.1+batch20251228.donor42.wildtype
2.3.1+batch20251228.donor42.gfp-tagged
\end{verbatim}

\hypertarget{compatibility-declarations}{%
\subsubsection{Compatibility
Declarations}\label{compatibility-declarations}}

\begin{Shaded}
\begin{Highlighting}[]
\FunctionTok{compatibility}\KeywordTok{:}
\AttributeTok{  }\FunctionTok{backward\_compatible\_with}\KeywordTok{:}\AttributeTok{ }\KeywordTok{[}\StringTok{"2.2.x"}\KeywordTok{,}\AttributeTok{ }\StringTok{"2.1.x"}\KeywordTok{]}
\AttributeTok{  }\FunctionTok{known\_incompatible}\KeywordTok{:}\AttributeTok{ }\KeywordTok{[}\StringTok{"1.x.x"}\KeywordTok{]}
\AttributeTok{  }
\AttributeTok{  }\FunctionTok{interface\_changes}\KeywordTok{:}
\AttributeTok{    }\KeywordTok{{-}}\AttributeTok{ }\FunctionTok{version}\KeywordTok{:}\AttributeTok{ }\StringTok{"2.0.0"}
\AttributeTok{      }\FunctionTok{change}\KeywordTok{:}\AttributeTok{ }\StringTok{"electrical input voltage range expanded"}
\AttributeTok{      }\FunctionTok{migration}\KeywordTok{:}\AttributeTok{ }\StringTok{"no action required for existing users"}
\end{Highlighting}
\end{Shaded}

\hypertarget{example-complete-muscle-actuator-specification}{%
\subsection{4.4 Example: Complete Muscle Actuator
Specification}\label{example-complete-muscle-actuator-specification}}

\begin{Shaded}
\begin{Highlighting}[]
\FunctionTok{bio{-}component}\KeywordTok{:}\AttributeTok{ }\StringTok{"1.0"}

\FunctionTok{info}\KeywordTok{:}
\AttributeTok{  }\FunctionTok{id}\KeywordTok{:}\AttributeTok{ }\StringTok{"muscle{-}actuator{-}human{-}skeletal"}
\AttributeTok{  }\FunctionTok{name}\KeywordTok{:}\AttributeTok{ }\StringTok{"Human Skeletal Muscle Actuator"}
\AttributeTok{  }\FunctionTok{version}\KeywordTok{:}\AttributeTok{ }\StringTok{"2.3.1"}
\AttributeTok{  }\FunctionTok{description}\KeywordTok{:}\AttributeTok{ }\StringTok{"Contractile muscle tissue for force generation"}
\AttributeTok{  }\FunctionTok{license}\KeywordTok{:}\AttributeTok{ }\StringTok{"CC{-}BY{-}SA{-}4.0"}
\AttributeTok{  }\FunctionTok{authors}\KeywordTok{:}\AttributeTok{ }\KeywordTok{[}\StringTok{"Wetware Engineering Project"}\KeywordTok{]}
\AttributeTok{  }\FunctionTok{repository}\KeywordTok{:}\AttributeTok{ }\StringTok{"https://github.com/wetware/muscle{-}actuator"}

\FunctionTok{classification}\KeywordTok{:}
\AttributeTok{  }\FunctionTok{type}\KeywordTok{:}\AttributeTok{ }\StringTok{"actuator"}
\AttributeTok{  }\FunctionTok{domain}\KeywordTok{:}\AttributeTok{ }\StringTok{"musculoskeletal"}
\AttributeTok{  }\FunctionTok{tags}\KeywordTok{:}\AttributeTok{ }\KeywordTok{[}\StringTok{"muscle"}\KeywordTok{,}\AttributeTok{ }\StringTok{"contractile"}\KeywordTok{,}\AttributeTok{ }\StringTok{"force{-}generation"}\KeywordTok{,}\AttributeTok{ }\StringTok{"human"}\KeywordTok{]}

\FunctionTok{source}\KeywordTok{:}
\AttributeTok{  }\FunctionTok{organism}\KeywordTok{:}\AttributeTok{ }\StringTok{"Homo sapiens"}
\AttributeTok{  }\FunctionTok{tissue\_type}\KeywordTok{:}\AttributeTok{ }\StringTok{"skeletal muscle"}
\AttributeTok{  }\FunctionTok{cell\_types}\KeywordTok{:}\AttributeTok{ }\KeywordTok{[}\StringTok{"myocyte"}\KeywordTok{,}\AttributeTok{ }\StringTok{"satellite cell"}\KeywordTok{]}
\AttributeTok{  }\FunctionTok{culture\_protocol}\KeywordTok{:}\AttributeTok{ }\StringTok{"https://protocols.io/wetware/muscle{-}culture{-}v2"}
\AttributeTok{  }\FunctionTok{biosafety\_level}\KeywordTok{:}\AttributeTok{ }\StringTok{"BSL{-}1"}

\FunctionTok{interface}\KeywordTok{:}
\AttributeTok{  }\FunctionTok{inputs}\KeywordTok{:}
\AttributeTok{    }\KeywordTok{{-}}\AttributeTok{ }\FunctionTok{id}\KeywordTok{:}\AttributeTok{ }\StringTok{"electrical\_stimulation"}
\AttributeTok{      }\FunctionTok{type}\KeywordTok{:}\AttributeTok{ }\StringTok{"electrical"}
\AttributeTok{      }\FunctionTok{required}\KeywordTok{:}\AttributeTok{ }\CharTok{true}
\AttributeTok{      }\FunctionTok{parameters}\KeywordTok{:}
\AttributeTok{        }\FunctionTok{voltage}\KeywordTok{:}\AttributeTok{ }\KeywordTok{\{}\AttributeTok{ }\FunctionTok{range}\KeywordTok{:}\AttributeTok{ }\KeywordTok{[}\DecValTok{0}\KeywordTok{,}\AttributeTok{ }\DecValTok{5}\KeywordTok{],}\AttributeTok{ }\FunctionTok{unit}\KeywordTok{:}\AttributeTok{ }\StringTok{"V"}\AttributeTok{ }\KeywordTok{\}}
\AttributeTok{        }\FunctionTok{frequency}\KeywordTok{:}\AttributeTok{ }\KeywordTok{\{}\AttributeTok{ }\FunctionTok{range}\KeywordTok{:}\AttributeTok{ }\KeywordTok{[}\DecValTok{1}\KeywordTok{,}\AttributeTok{ }\DecValTok{100}\KeywordTok{],}\AttributeTok{ }\FunctionTok{unit}\KeywordTok{:}\AttributeTok{ }\StringTok{"Hz"}\AttributeTok{ }\KeywordTok{\}}
\AttributeTok{        }\FunctionTok{pulse\_width}\KeywordTok{:}\AttributeTok{ }\KeywordTok{\{}\AttributeTok{ }\FunctionTok{range}\KeywordTok{:}\AttributeTok{ }\KeywordTok{[}\FloatTok{0.1}\KeywordTok{,}\AttributeTok{ }\DecValTok{10}\KeywordTok{],}\AttributeTok{ }\FunctionTok{unit}\KeywordTok{:}\AttributeTok{ }\StringTok{"ms"}\AttributeTok{ }\KeywordTok{\}}
\AttributeTok{      }\FunctionTok{response\_time}\KeywordTok{:}\AttributeTok{ }\DecValTok{50}\CommentTok{  \# ms}
\AttributeTok{    }
\AttributeTok{    }\KeywordTok{{-}}\AttributeTok{ }\FunctionTok{id}\KeywordTok{:}\AttributeTok{ }\StringTok{"perfusion\_input"}
\AttributeTok{      }\FunctionTok{type}\KeywordTok{:}\AttributeTok{ }\StringTok{"perfusion"}
\AttributeTok{      }\FunctionTok{required}\KeywordTok{:}\AttributeTok{ }\CharTok{true}
\AttributeTok{      }\FunctionTok{parameters}\KeywordTok{:}
\AttributeTok{        }\FunctionTok{flow\_rate}\KeywordTok{:}\AttributeTok{ }\KeywordTok{\{}\AttributeTok{ }\FunctionTok{range}\KeywordTok{:}\AttributeTok{ }\KeywordTok{[}\FloatTok{0.1}\KeywordTok{,}\AttributeTok{ }\FloatTok{2.0}\KeywordTok{],}\AttributeTok{ }\FunctionTok{unit}\KeywordTok{:}\AttributeTok{ }\StringTok{"mL/min"}\AttributeTok{ }\KeywordTok{\}}
\AttributeTok{        }
\AttributeTok{  }\FunctionTok{outputs}\KeywordTok{:}
\AttributeTok{    }\KeywordTok{{-}}\AttributeTok{ }\FunctionTok{id}\KeywordTok{:}\AttributeTok{ }\StringTok{"force\_output"}
\AttributeTok{      }\FunctionTok{type}\KeywordTok{:}\AttributeTok{ }\StringTok{"mechanical"}
\AttributeTok{      }\FunctionTok{parameters}\KeywordTok{:}
\AttributeTok{        }\FunctionTok{force}\KeywordTok{:}\AttributeTok{ }\KeywordTok{\{}\AttributeTok{ }\FunctionTok{range}\KeywordTok{:}\AttributeTok{ }\KeywordTok{[}\DecValTok{0}\KeywordTok{,}\AttributeTok{ }\DecValTok{50}\KeywordTok{],}\AttributeTok{ }\FunctionTok{unit}\KeywordTok{:}\AttributeTok{ }\StringTok{"mN"}\AttributeTok{ }\KeywordTok{\}}
\AttributeTok{        }\FunctionTok{displacement}\KeywordTok{:}\AttributeTok{ }\KeywordTok{\{}\AttributeTok{ }\FunctionTok{range}\KeywordTok{:}\AttributeTok{ }\KeywordTok{[}\DecValTok{0}\KeywordTok{,}\AttributeTok{ }\DecValTok{5}\KeywordTok{],}\AttributeTok{ }\FunctionTok{unit}\KeywordTok{:}\AttributeTok{ }\StringTok{"mm"}\AttributeTok{ }\KeywordTok{\}}
\AttributeTok{      }\FunctionTok{latency}\KeywordTok{:}\AttributeTok{ }\DecValTok{150}\CommentTok{  \# ms}
\AttributeTok{      }\FunctionTok{monitoring}\KeywordTok{:}
\AttributeTok{        }\FunctionTok{metrics}\KeywordTok{:}\AttributeTok{ }\KeywordTok{[}\StringTok{"force"}\KeywordTok{,}\AttributeTok{ }\StringTok{"displacement"}\KeywordTok{,}\AttributeTok{ }\StringTok{"velocity"}\KeywordTok{]}
\AttributeTok{        }\FunctionTok{sampling\_rate}\KeywordTok{:}\AttributeTok{ }\DecValTok{100}\CommentTok{  \# Hz}

\FunctionTok{requirements}\KeywordTok{:}
\AttributeTok{  }\FunctionTok{physical}\KeywordTok{:}
\AttributeTok{    }\FunctionTok{temperature}\KeywordTok{:}\AttributeTok{ }\KeywordTok{\{}\AttributeTok{ }\FunctionTok{optimal}\KeywordTok{:}\AttributeTok{ }\DecValTok{37}\KeywordTok{,}\AttributeTok{ }\FunctionTok{range}\KeywordTok{:}\AttributeTok{ }\KeywordTok{[}\DecValTok{35}\KeywordTok{,}\AttributeTok{ }\DecValTok{39}\KeywordTok{],}\AttributeTok{ }\FunctionTok{unit}\KeywordTok{:}\AttributeTok{ }\StringTok{"°C"}\AttributeTok{ }\KeywordTok{\}}
\AttributeTok{  }\FunctionTok{chemical}\KeywordTok{:}
\AttributeTok{    }\FunctionTok{pH}\KeywordTok{:}\AttributeTok{ }\KeywordTok{\{}\AttributeTok{ }\FunctionTok{optimal}\KeywordTok{:}\AttributeTok{ }\FloatTok{7.4}\KeywordTok{,}\AttributeTok{ }\FunctionTok{range}\KeywordTok{:}\AttributeTok{ }\KeywordTok{[}\FloatTok{7.2}\KeywordTok{,}\AttributeTok{ }\FloatTok{7.6}\KeywordTok{]}\AttributeTok{ }\KeywordTok{\}}
\AttributeTok{    }\FunctionTok{oxygen}\KeywordTok{:}\AttributeTok{ }\KeywordTok{\{}\AttributeTok{ }\FunctionTok{range}\KeywordTok{:}\AttributeTok{ }\KeywordTok{[}\DecValTok{15}\KeywordTok{,}\AttributeTok{ }\DecValTok{25}\KeywordTok{],}\AttributeTok{ }\FunctionTok{unit}\KeywordTok{:}\AttributeTok{ }\StringTok{"\%"}\AttributeTok{ }\KeywordTok{\}}
\AttributeTok{    }\FunctionTok{glucose}\KeywordTok{:}\AttributeTok{ }\KeywordTok{\{}\AttributeTok{ }\FunctionTok{range}\KeywordTok{:}\AttributeTok{ }\KeywordTok{[}\DecValTok{5}\KeywordTok{,}\AttributeTok{ }\DecValTok{25}\KeywordTok{],}\AttributeTok{ }\FunctionTok{unit}\KeywordTok{:}\AttributeTok{ }\StringTok{"mM"}\AttributeTok{ }\KeywordTok{\}}

\FunctionTok{performance}\KeywordTok{:}
\AttributeTok{  }\FunctionTok{functional}\KeywordTok{:}
\AttributeTok{    }\FunctionTok{max\_force}\KeywordTok{:}\AttributeTok{ }\KeywordTok{\{}\AttributeTok{ }\FunctionTok{value}\KeywordTok{:}\AttributeTok{ }\DecValTok{50}\KeywordTok{,}\AttributeTok{ }\FunctionTok{unit}\KeywordTok{:}\AttributeTok{ }\StringTok{"mN"}\AttributeTok{ }\KeywordTok{\}}
\AttributeTok{    }\FunctionTok{response\_time}\KeywordTok{:}\AttributeTok{ }\KeywordTok{\{}\AttributeTok{ }\FunctionTok{typical}\KeywordTok{:}\AttributeTok{ }\DecValTok{150}\KeywordTok{,}\AttributeTok{ }\FunctionTok{max}\KeywordTok{:}\AttributeTok{ }\DecValTok{300}\KeywordTok{,}\AttributeTok{ }\FunctionTok{unit}\KeywordTok{:}\AttributeTok{ }\StringTok{"ms"}\AttributeTok{ }\KeywordTok{\}}
\AttributeTok{    }\FunctionTok{contraction\_velocity}\KeywordTok{:}\AttributeTok{ }\KeywordTok{\{}\AttributeTok{ }\FunctionTok{max}\KeywordTok{:}\AttributeTok{ }\DecValTok{10}\KeywordTok{,}\AttributeTok{ }\FunctionTok{unit}\KeywordTok{:}\AttributeTok{ }\StringTok{"mm/s"}\AttributeTok{ }\KeywordTok{\}}
\AttributeTok{  }\FunctionTok{reliability}\KeywordTok{:}
\AttributeTok{    }\FunctionTok{lifetime}\KeywordTok{:}\AttributeTok{ }\KeywordTok{\{}\AttributeTok{ }\FunctionTok{mean}\KeywordTok{:}\AttributeTok{ }\DecValTok{14}\KeywordTok{,}\AttributeTok{ }\FunctionTok{std}\KeywordTok{:}\AttributeTok{ }\DecValTok{3}\KeywordTok{,}\AttributeTok{ }\FunctionTok{unit}\KeywordTok{:}\AttributeTok{ }\StringTok{"days"}\AttributeTok{ }\KeywordTok{\}}
\AttributeTok{    }\FunctionTok{failure\_rate}\KeywordTok{:}\AttributeTok{ }\KeywordTok{\{}\AttributeTok{ }\FunctionTok{value}\KeywordTok{:}\AttributeTok{ }\FloatTok{0.01}\KeywordTok{,}\AttributeTok{ }\FunctionTok{unit}\KeywordTok{:}\AttributeTok{ }\StringTok{"per\_hour"}\AttributeTok{ }\KeywordTok{\}}
\AttributeTok{  }\FunctionTok{resources}\KeywordTok{:}
\AttributeTok{    }\FunctionTok{oxygen\_consumption}\KeywordTok{:}\AttributeTok{ }\KeywordTok{\{}\AttributeTok{ }\FunctionTok{value}\KeywordTok{:}\AttributeTok{ }\DecValTok{5}\KeywordTok{,}\AttributeTok{ }\FunctionTok{unit}\KeywordTok{:}\AttributeTok{ }\StringTok{"µmol/hour"}\AttributeTok{ }\KeywordTok{\}}
\AttributeTok{    }\FunctionTok{glucose\_consumption}\KeywordTok{:}\AttributeTok{ }\KeywordTok{\{}\AttributeTok{ }\FunctionTok{value}\KeywordTok{:}\AttributeTok{ }\FloatTok{2.5}\KeywordTok{,}\AttributeTok{ }\FunctionTok{unit}\KeywordTok{:}\AttributeTok{ }\StringTok{"µmol/hour"}\AttributeTok{ }\KeywordTok{\}}

\FunctionTok{failure\_modes}\KeywordTok{:}
\AttributeTok{  }\KeywordTok{{-}}\AttributeTok{ }\FunctionTok{id}\KeywordTok{:}\AttributeTok{ }\StringTok{"fatigue"}
\AttributeTok{    }\FunctionTok{type}\KeywordTok{:}\AttributeTok{ }\StringTok{"recoverable"}
\AttributeTok{    }\FunctionTok{probability}\KeywordTok{:}\AttributeTok{ }\FloatTok{0.3}
\AttributeTok{    }\FunctionTok{detection}\KeywordTok{:}\AttributeTok{ }\StringTok{"force\_output \textless{} 80\% baseline"}
\AttributeTok{    }\FunctionTok{impact}\KeywordTok{:}\AttributeTok{ }\StringTok{"reduced\_performance"}
\AttributeTok{    }\FunctionTok{mitigation}\KeywordTok{:}\AttributeTok{ }\StringTok{"reduce stimulation, allow recovery"}
\AttributeTok{  }
\AttributeTok{  }\KeywordTok{{-}}\AttributeTok{ }\FunctionTok{id}\KeywordTok{:}\AttributeTok{ }\StringTok{"necrosis"}
\AttributeTok{    }\FunctionTok{type}\KeywordTok{:}\AttributeTok{ }\StringTok{"irreversible"}
\AttributeTok{    }\FunctionTok{probability}\KeywordTok{:}\AttributeTok{ }\FloatTok{0.05}
\AttributeTok{    }\FunctionTok{detection}\KeywordTok{:}\AttributeTok{ }\StringTok{"viability \textless{} 50\%"}
\AttributeTok{    }\FunctionTok{impact}\KeywordTok{:}\AttributeTok{ }\StringTok{"component\_loss"}
\AttributeTok{    }\FunctionTok{mitigation}\KeywordTok{:}\AttributeTok{ }\StringTok{"replace component"}

\FunctionTok{testing}\KeywordTok{:}
\AttributeTok{  }\FunctionTok{unit\_tests}\KeywordTok{:}
\AttributeTok{    }\KeywordTok{{-}}\AttributeTok{ }\FunctionTok{id}\KeywordTok{:}\AttributeTok{ }\StringTok{"contraction\_response"}
\AttributeTok{      }\FunctionTok{description}\KeywordTok{:}\AttributeTok{ }\StringTok{"Verify force output on stimulation"}
\AttributeTok{      }\FunctionTok{protocol}\KeywordTok{:}\AttributeTok{ }\StringTok{"stimulate 10Hz 2V for 1s, measure force"}
\AttributeTok{      }\FunctionTok{acceptance}\KeywordTok{:}\AttributeTok{ }\StringTok{"force in [5, 15] mN within 200ms"}
\AttributeTok{    }
\AttributeTok{    }\KeywordTok{{-}}\AttributeTok{ }\FunctionTok{id}\KeywordTok{:}\AttributeTok{ }\StringTok{"viability\_check"}
\AttributeTok{      }\FunctionTok{description}\KeywordTok{:}\AttributeTok{ }\StringTok{"Verify cell survival"}
\AttributeTok{      }\FunctionTok{protocol}\KeywordTok{:}\AttributeTok{ }\StringTok{"live/dead staining"}
\AttributeTok{      }\FunctionTok{acceptance}\KeywordTok{:}\AttributeTok{ }\StringTok{"viability \textgreater{} 90\%"}

\FunctionTok{dependencies}\KeywordTok{:}
\AttributeTok{  }\FunctionTok{adapters}\KeywordTok{:}
\AttributeTok{    }\KeywordTok{{-}}\AttributeTok{ }\FunctionTok{id}\KeywordTok{:}\AttributeTok{ }\StringTok{"perfusion{-}adapter"}
\AttributeTok{      }\FunctionTok{version}\KeywordTok{:}\AttributeTok{ }\StringTok{"\^{}1.0.0"}
\AttributeTok{  }\FunctionTok{protocols}\KeywordTok{:}
\AttributeTok{    }\KeywordTok{{-}}\AttributeTok{ }\FunctionTok{name}\KeywordTok{:}\AttributeTok{ }\StringTok{"standard{-}mammalian{-}culture"}
\AttributeTok{      }\FunctionTok{version}\KeywordTok{:}\AttributeTok{ }\StringTok{"2.1"}
\end{Highlighting}
\end{Shaded}

\hypertarget{bio-dsl-language-design-rationale}{%
\section{5. Bio-DSL: Language Design
Rationale}\label{bio-dsl-language-design-rationale}}

\hypertarget{why-a-domain-specific-language}{%
\subsection{5.1 Why a Domain-Specific
Language?}\label{why-a-domain-specific-language}}

Martin Fowler defines a domain-specific language (DSL) as ``a computer
language specialized to a particular application domain.'' DSLs trade
generality for expressiveness within their domain.

\hypertarget{benefits-of-dsls}{%
\subsubsection{Benefits of DSLs}\label{benefits-of-dsls}}

\begin{longtable}[]{@{}
  >{\raggedright\arraybackslash}p{(\columnwidth - 4\tabcolsep) * \real{0.2093}}
  >{\raggedright\arraybackslash}p{(\columnwidth - 4\tabcolsep) * \real{0.3023}}
  >{\raggedright\arraybackslash}p{(\columnwidth - 4\tabcolsep) * \real{0.4884}}@{}}
\toprule\noalign{}
\begin{minipage}[b]{\linewidth}\raggedright
Benefit
\end{minipage} & \begin{minipage}[b]{\linewidth}\raggedright
Explanation
\end{minipage} & \begin{minipage}[b]{\linewidth}\raggedright
Bio-DSL Application
\end{minipage} \\
\midrule\noalign{}
\endhead
\bottomrule\noalign{}
\endlastfoot
\textbf{Expressiveness} & Say more with less & Describe complex
assemblies concisely \\
\textbf{Readability} & Domain experts can understand & Biologists can
read system descriptions \\
\textbf{Validation} & Domain-specific error checking & Catch interface
mismatches at ``compile time'' \\
\textbf{Abstraction} & Hide implementation details & Focus on what, not
how \\
\end{longtable}

\hypertarget{why-not-use-existing-languages}{%
\subsubsection{Why Not Use Existing
Languages?}\label{why-not-use-existing-languages}}

General-purpose languages (Python, JavaScript) could describe biological
systems, but: - Too much flexibility allows invalid configurations - No
built-in understanding of biological constraints - Error messages would
be generic, not domain-specific

Existing biological languages (SBML, SBOL) operate at different
abstraction levels: - SBML: molecular reaction networks - SBOL: genetic
sequences - Bio-DSL: organ/system-level composition

\hypertarget{design-goals}{%
\subsection{5.2 Design Goals}\label{design-goals}}

\begin{enumerate}
\def\labelenumi{\arabic{enumi}.}
\tightlist
\item
  \textbf{Declarative}: Describe \emph{what} the system is, not
  \emph{how} to build it
\item
  \textbf{Readable}: A biologist should understand the intent without
  programming background
\item
  \textbf{Verifiable}: Static analysis can catch errors before physical
  assembly
\item
  \textbf{Executable}: Can generate runtime configurations and
  monitoring dashboards
\item
  \textbf{Composable}: Systems can be nested and reused
\end{enumerate}

\hypertarget{language-constructs}{%
\subsection{5.3 Language Constructs}\label{language-constructs}}

\hypertarget{component-declaration}{%
\subsubsection{Component Declaration}\label{component-declaration}}

\begin{Shaded}
\begin{Highlighting}[]
\NormalTok{// Import component from registry with version constraint}
\NormalTok{COMPONENT \textless{}alias\textgreater{} FROM "\textless{}source\textgreater{}@\textless{}version\textgreater{}" [AS \textless{}local\_name\textgreater{}]}

\NormalTok{// Examples:}
\NormalTok{COMPONENT flexor FROM "muscle{-}actuator{-}human{-}skeletal@\^{}2.3"}
\NormalTok{COMPONENT sensor FROM "piezo{-}force{-}sensor@\textasciitilde{}1.1" AS force\_sensor}
\NormalTok{COMPONENT controller FROM "neural{-}organoid{-}spinal@\textgreater{}=0.8"}
\end{Highlighting}
\end{Shaded}

Version constraints follow npm conventions: - \texttt{\^{}2.3.0}:
Compatible with 2.x.x (\textgreater=2.3.0 \textless3.0.0) -
\texttt{\textasciitilde{}1.1.0}: Approximately 1.1.x (\textgreater=1.1.0
\textless1.2.0) - \texttt{\textgreater{}=0.8}: At least version 0.8

\hypertarget{connection-declaration}{%
\subsubsection{Connection Declaration}\label{connection-declaration}}

\begin{Shaded}
\begin{Highlighting}[]
\NormalTok{// Basic connection}
\NormalTok{CONNECT \textless{}source\textgreater{}.\textless{}port\textgreater{} TO \textless{}target\textgreater{}.\textless{}port\textgreater{}}

\NormalTok{// Connection with adapter}
\NormalTok{CONNECT \textless{}source\textgreater{}.\textless{}port\textgreater{} TO \textless{}target\textgreater{}.\textless{}port\textgreater{} VIA \textless{}adapter\textgreater{}}

\NormalTok{// Connection with parameters}
\NormalTok{CONNECT \textless{}source\textgreater{}.\textless{}port\textgreater{} TO \textless{}target\textgreater{}.\textless{}port\textgreater{} WITH \{ \textless{}params\textgreater{} \}}

\NormalTok{// Examples:}
\NormalTok{CONNECT sensor.output TO controller.input}
\NormalTok{CONNECT controller.output TO flexor.stimulation VIA signal\_converter}
\NormalTok{CONNECT flexor.force\_output TO sensor.input WITH \{ gain: 1.5, offset: 0 \}}
\end{Highlighting}
\end{Shaded}

\hypertarget{adapter-declaration}{%
\subsubsection{Adapter Declaration}\label{adapter-declaration}}

\begin{Shaded}
\begin{Highlighting}[]
\NormalTok{// Declare adapter for interface conversion}
\NormalTok{ADAPTER \textless{}alias\textgreater{} FROM "\textless{}source\textgreater{}@\textless{}version\textgreater{}" [WITH \{ \textless{}config\textgreater{} \}]}

\NormalTok{// Examples:}
\NormalTok{ADAPTER signal\_converter FROM "opto{-}electrical@2.0" WITH \{}
\NormalTok{  input\_type: "electrical",}
\NormalTok{  output\_type: "optical",}
\NormalTok{  wavelength: 470 nm}
\NormalTok{\}}

\NormalTok{ADAPTER perfusion\_hub FROM "microfluidic{-}4ch@1.0" WITH \{}
\NormalTok{  channels: 4,}
\NormalTok{  flow\_rate: 0.5 mL/min}
\NormalTok{\}}
\end{Highlighting}
\end{Shaded}

\hypertarget{runtime-configuration}{%
\subsubsection{Runtime Configuration}\label{runtime-configuration}}

\begin{Shaded}
\begin{Highlighting}[]
\NormalTok{RUNTIME \{}
\NormalTok{  // Perfusion settings}
\NormalTok{  perfusion: \{}
\NormalTok{    medium: "DMEM + 10\% FBS",}
\NormalTok{    temperature: 37 °C,}
\NormalTok{    pH: 7.4,}
\NormalTok{    flow\_rate: 0.5 mL/min,}
\NormalTok{    oxygenation: true}
\NormalTok{  \},}
  
\NormalTok{  // Control settings}
\NormalTok{  control: \{}
\NormalTok{    mode: "closed\_loop",}
\NormalTok{    algorithm: "PID",}
\NormalTok{    parameters: \{ Kp: 0.8, Ki: 0.2, Kd: 0.1 \}}
\NormalTok{  \},}
  
\NormalTok{  // Monitoring settings}
\NormalTok{  monitoring: \{}
\NormalTok{    log\_interval: 10 s,}
\NormalTok{    metrics: ["force", "viability", "temperature"],}
\NormalTok{    alerts: \{}
\NormalTok{      "viability \textless{} 80\%": "WARNING",}
\NormalTok{      "temperature \textgreater{} 39°C": "CRITICAL"}
\NormalTok{    \}}
\NormalTok{  \},}
  
\NormalTok{  // Safety settings}
\NormalTok{  safety: \{}
\NormalTok{    emergency\_stop: "viability \textless{} 50\%",}
\NormalTok{    max\_force: 100 mN}
\NormalTok{  \}}
\NormalTok{\}}
\end{Highlighting}
\end{Shaded}

\hypertarget{behavioral-logic}{%
\subsubsection{Behavioral Logic}\label{behavioral-logic}}

\begin{Shaded}
\begin{Highlighting}[]
\NormalTok{// Event{-}triggered actions}
\NormalTok{ON \textless{}event\textgreater{} DO \{ \textless{}actions\textgreater{} \}}

\NormalTok{// Condition{-}triggered actions}
\NormalTok{WHEN \textless{}condition\textgreater{} THEN \{ \textless{}actions\textgreater{} \}}

\NormalTok{// Periodic actions}
\NormalTok{EVERY \textless{}interval\textgreater{} DO \{ \textless{}actions\textgreater{} \}}

\NormalTok{// Examples:}
\NormalTok{ON STARTUP DO \{}
\NormalTok{  SET perfusion.flow\_rate = 0.5 mL/min}
\NormalTok{  WAIT 300 s  // Equilibration}
\NormalTok{  SET controller.mode = "active"}
\NormalTok{\}}

\NormalTok{WHEN flexor.fatigue\_index \textgreater{} 0.3 THEN \{}
\NormalTok{  LOG "Fatigue detected"}
\NormalTok{  REDUCE flexor.stimulation\_frequency BY 20\%}
\NormalTok{\}}

\NormalTok{EVERY 1 hour DO \{}
\NormalTok{  RUN viability\_check()}
\NormalTok{  IF any.viability \textless{} 85\% THEN \{}
\NormalTok{    INCREASE perfusion.flow\_rate BY 10\%}
\NormalTok{  \}}
\NormalTok{\}}
\end{Highlighting}
\end{Shaded}

\hypertarget{test-declaration}{%
\subsubsection{Test Declaration}\label{test-declaration}}

\begin{Shaded}
\begin{Highlighting}[]
\NormalTok{TEST \textless{}name\textgreater{} \{}
\NormalTok{  description: "\textless{}text\textgreater{}"}
  
\NormalTok{  GIVEN \textless{}preconditions\textgreater{}}
\NormalTok{  WHEN \textless{}actions\textgreater{}}
\NormalTok{  THEN \textless{}assertions\textgreater{}}
\NormalTok{\}}

\NormalTok{// Example:}
\NormalTok{TEST contraction\_response \{}
\NormalTok{  description: "Verify muscle responds to stimulation"}
  
\NormalTok{  GIVEN flexor.state == "ready"}
\NormalTok{  WHEN STIMULATE flexor AT 10 Hz, 2 V FOR 1 s}
\NormalTok{  THEN EXPECT flexor.force IN [5, 15] mN WITHIN 200 ms}
\NormalTok{\}}

\NormalTok{TEST closed\_loop\_tracking \{}
\NormalTok{  description: "Verify feedback control accuracy"}
  
\NormalTok{  GIVEN system.mode == "closed\_loop"}
\NormalTok{  WHEN SET target = sine\_wave(0.5 Hz, 20 mN) FOR 60 s}
\NormalTok{  THEN EXPECT tracking\_error \textless{} 3 mN RMS}
\NormalTok{\}}
\end{Highlighting}
\end{Shaded}

\hypertarget{complete-example-dual-muscle-antagonist-system}{%
\subsection{5.4 Complete Example: Dual-Muscle Antagonist
System}\label{complete-example-dual-muscle-antagonist-system}}

\begin{Shaded}
\begin{Highlighting}[]
\NormalTok{// ============================================}
\NormalTok{// Bio{-}Mechanical Arm Unit v0.1}
\NormalTok{// Dual{-}muscle antagonist with closed{-}loop control}
\NormalTok{// ============================================}

\NormalTok{// === Component Declarations ===}
\NormalTok{COMPONENT flexor FROM "muscle{-}actuator{-}human{-}skeletal@\^{}2.3" \{}
\NormalTok{  role: "agonist",}
\NormalTok{  force\_range: [0, 50] mN}
\NormalTok{\}}

\NormalTok{COMPONENT extensor FROM "muscle{-}actuator{-}human{-}skeletal@\^{}2.3" \{}
\NormalTok{  role: "antagonist", }
\NormalTok{  force\_range: [0, 50] mN}
\NormalTok{\}}

\NormalTok{COMPONENT sensor FROM "piezo{-}force{-}sensor@\textasciitilde{}1.1" \{}
\NormalTok{  range: [0, 100] mN,}
\NormalTok{  sampling\_rate: 100 Hz}
\NormalTok{\}}

\NormalTok{COMPONENT controller FROM "neural{-}organoid{-}spinal@\textgreater{}=0.8" \{}
\NormalTok{  input\_channels: 2,}
\NormalTok{  output\_channels: 2}
\NormalTok{\}}

\NormalTok{// === Adapter Declarations ===}
\NormalTok{ADAPTER perfusion FROM "microfluidic{-}4ch@1.0" \{}
\NormalTok{  medium: "DMEM + 10\% FBS",}
\NormalTok{  flow\_rate: 0.5 mL/min PER channel}
\NormalTok{\}}

\NormalTok{ADAPTER stim\_converter FROM "opto{-}electrical@2.0" \{}
\NormalTok{  wavelength: 470 nm}
\NormalTok{\}}

\NormalTok{// === Connection Topology ===}

\NormalTok{// Controller to muscles (via optical stimulation)}
\NormalTok{CONNECT controller.output\_1 TO flexor.stimulation }
\NormalTok{  VIA stim\_converter}
\NormalTok{  WITH \{ frequency: [1, 50] Hz, voltage: [0, 3] V \}}

\NormalTok{CONNECT controller.output\_2 TO extensor.stimulation}
\NormalTok{  VIA stim\_converter}
\NormalTok{  WITH \{ frequency: [1, 50] Hz, voltage: [0, 3] V \}}

\NormalTok{// Sensor feedback to controller}
\NormalTok{CONNECT sensor.output TO controller.feedback\_input}
\NormalTok{  WITH \{ gain: 1.5 \}}

\NormalTok{// Perfusion connections}
\NormalTok{CONNECT perfusion.ch1 TO flexor.perfusion\_input}
\NormalTok{CONNECT perfusion.ch2 TO extensor.perfusion\_input}
\NormalTok{CONNECT perfusion.ch3 TO controller.perfusion\_input}
\NormalTok{CONNECT perfusion.ch4 TO sensor.perfusion\_input}

\NormalTok{// === Runtime Configuration ===}
\NormalTok{RUNTIME \{}
\NormalTok{  perfusion: \{}
\NormalTok{    temperature: 37 °C,}
\NormalTok{    pH: 7.4,}
\NormalTok{    oxygenation: true,}
\NormalTok{    waste\_removal: "continuous"}
\NormalTok{  \},}
  
\NormalTok{  control: \{}
\NormalTok{    mode: "closed\_loop",}
\NormalTok{    target: "position",}
\NormalTok{    pid: \{ Kp: 0.8, Ki: 0.2, Kd: 0.1 \}}
\NormalTok{  \},}
  
\NormalTok{  monitoring: \{}
\NormalTok{    interval: 10 s,}
\NormalTok{    metrics: [}
\NormalTok{      "flexor.force", "extensor.force",}
\NormalTok{      "sensor.reading", "controller.activity",}
\NormalTok{      "*.viability"}
\NormalTok{    ],}
\NormalTok{    alerts: \{}
\NormalTok{      "viability \textless{} 80\%": "WARNING",}
\NormalTok{      "force \textgreater{} 90 mN": "CRITICAL"}
\NormalTok{    \}}
\NormalTok{  \},}
  
\NormalTok{  safety: \{}
\NormalTok{    max\_force: 100 mN,}
\NormalTok{    emergency: \{}
\NormalTok{      trigger: "viability \textless{} 50\% OR force \textgreater{} 120 mN",}
\NormalTok{      action: "STOP\_ALL; PERFUSION\_ONLY"}
\NormalTok{    \}}
\NormalTok{  \}}
\NormalTok{\}}

\NormalTok{// === Behavioral Logic ===}
\NormalTok{ON STARTUP DO \{}
\NormalTok{  SET perfusion.flow\_rate = 0.5 mL/min}
\NormalTok{  WAIT 300 s  // Equilibration period}
\NormalTok{  RUN calibration\_sequence()}
\NormalTok{  SET controller.mode = "active"}
\NormalTok{  LOG "System initialized"}
\NormalTok{\}}

\NormalTok{WHEN target\_position CHANGES THEN \{}
\NormalTok{  error = target\_position {-} sensor.reading}
\NormalTok{  controller.compute(error)}
\NormalTok{\}}

\NormalTok{WHEN flexor.fatigue\_index \textgreater{} 0.3 THEN \{}
\NormalTok{  LOG "Flexor fatigue detected"}
\NormalTok{  REDUCE flexor.stim\_frequency BY 20\%}
\NormalTok{  INCREASE extensor.stim\_frequency BY 10\%  // Compensate}
\NormalTok{\}}

\NormalTok{EVERY 1 hour DO \{}
\NormalTok{  RUN viability\_check()}
\NormalTok{  RECORD performance\_snapshot()}
\NormalTok{\}}

\NormalTok{// === Test Suite ===}
\NormalTok{TEST unit\_response \{}
\NormalTok{  description: "Single muscle contraction test"}
\NormalTok{  GIVEN flexor.state == "ready"}
\NormalTok{  WHEN STIMULATE flexor AT 10 Hz, 2 V FOR 1 s}
\NormalTok{  THEN EXPECT flexor.force IN [5, 15] mN WITHIN 200 ms}
\NormalTok{\}}

\NormalTok{TEST antagonist\_balance \{}
\NormalTok{  description: "Antagonist coordination test"}
\NormalTok{  GIVEN system.mode == "active"}
\NormalTok{  WHEN ACTIVATE flexor AND extensor SIMULTANEOUSLY}
\NormalTok{  THEN EXPECT |flexor.force {-} extensor.force| \textless{} 5 mN}
\NormalTok{\}}

\NormalTok{TEST tracking\_accuracy \{}
\NormalTok{  description: "Closed{-}loop tracking test"}
\NormalTok{  GIVEN system.mode == "closed\_loop"}
\NormalTok{  WHEN SET target = sine\_wave(0.5 Hz, ±20 mN) FOR 60 s}
\NormalTok{  THEN EXPECT tracking\_error \textless{} 3 mN RMS}
\NormalTok{\}}

\NormalTok{// === Expected Performance ===}
\NormalTok{EXPECTED \{}
\NormalTok{  position\_accuracy: ±2 mN,}
\NormalTok{  bandwidth: [0, 2] Hz,}
\NormalTok{  lifetime: [7, 14] days,}
\NormalTok{  power\_consumption: [50, 100] mW}
\NormalTok{\}}
\end{Highlighting}
\end{Shaded}

\hypertarget{comparison-with-related-languages}{%
\subsection{5.5 Comparison with Related
Languages}\label{comparison-with-related-languages}}

\begin{longtable}[]{@{}
  >{\raggedright\arraybackslash}p{(\columnwidth - 6\tabcolsep) * \real{0.1639}}
  >{\raggedright\arraybackslash}p{(\columnwidth - 6\tabcolsep) * \real{0.2951}}
  >{\raggedright\arraybackslash}p{(\columnwidth - 6\tabcolsep) * \real{0.1475}}
  >{\raggedright\arraybackslash}p{(\columnwidth - 6\tabcolsep) * \real{0.3934}}@{}}
\toprule\noalign{}
\begin{minipage}[b]{\linewidth}\raggedright
Language
\end{minipage} & \begin{minipage}[b]{\linewidth}\raggedright
Abstraction Level
\end{minipage} & \begin{minipage}[b]{\linewidth}\raggedright
Purpose
\end{minipage} & \begin{minipage}[b]{\linewidth}\raggedright
Relationship to Bio-DSL
\end{minipage} \\
\midrule\noalign{}
\endhead
\bottomrule\noalign{}
\endlastfoot
\textbf{SBOL} & Genetic & DNA sequence description & Lower level;
describes component internals \\
\textbf{SBML} & Molecular & Biochemical reaction networks & Lower level;
models component dynamics \\
\textbf{CellML} & Cellular & Cell mathematical models & Lower level;
component behavior models \\
\textbf{Bio-DSL} & Organ/System & Component composition & Higher level;
system assembly \\
\end{longtable}

Bio-DSL is designed to \textbf{complement} these languages: - Use SBOL
to describe genetic modifications within a component - Use SBML to model
the biochemical behavior of a component - Use Bio-DSL to describe how
components connect into systems

\hypertarget{tooling-vision}{%
\subsection{5.6 Tooling Vision}\label{tooling-vision}}

A complete Bio-DSL ecosystem would include:

\begin{enumerate}
\def\labelenumi{\arabic{enumi}.}
\tightlist
\item
  \textbf{Parser/Validator}: Check syntax and semantic correctness
\item
  \textbf{Type Checker}: Verify interface compatibility
\item
  \textbf{Simulator}: Predict system behavior before physical assembly
\item
  \textbf{Code Generator}: Produce runtime configurations
\item
  \textbf{Visual Editor}: Drag-and-drop system design
\item
  \textbf{Package Manager}: Discover and install components
\item
  \textbf{Test Runner}: Execute test suites
\item
  \textbf{Documentation Generator}: Produce human-readable specs \# 6.
  Fundamental Differences and Open Challenges
\end{enumerate}

While the paradigm transfer from software to biological engineering is
powerful, fundamental differences between the domains create challenges
that require novel solutions beyond direct mapping.

\hypertarget{determinism-vs.-stochasticity}{%
\subsection{6.1 Determinism
vs.~Stochasticity}\label{determinism-vs.-stochasticity}}

\hypertarget{the-difference}{%
\subsubsection{The Difference}\label{the-difference}}

\begin{longtable}[]{@{}
  >{\raggedright\arraybackslash}p{(\columnwidth - 2\tabcolsep) * \real{0.5263}}
  >{\raggedright\arraybackslash}p{(\columnwidth - 2\tabcolsep) * \real{0.4737}}@{}}
\toprule\noalign{}
\begin{minipage}[b]{\linewidth}\raggedright
Software
\end{minipage} & \begin{minipage}[b]{\linewidth}\raggedright
Biology
\end{minipage} \\
\midrule\noalign{}
\endhead
\bottomrule\noalign{}
\endlastfoot
Function calls always return & Cells may die unexpectedly \\
Same input → same output & Biological variability is inherent \\
Errors can be precisely located & Failure modes are complex and
interacting \\
State is fully observable & Internal state is partially hidden \\
\end{longtable}

A software function \texttt{add(2,\ 3)} will always return \texttt{5}. A
biological muscle stimulated with identical parameters will produce
slightly different force each time, and occasionally may not respond at
all.

\hypertarget{implications-for-wetware-engineering}
\AttributeTok{    }\FunctionTok{failure\_probability}\KeywordTok{:}\AttributeTok{ }\FloatTok{0.01}
\end{Highlighting}
\end{Shaded}

\textbf{Testing must be statistical}:

\begin{Shaded}
\begin{Highlighting}[]
\FunctionTok{acceptance\_criteria}\KeywordTok{:}
\AttributeTok{  }\FunctionTok{metric}\KeywordTok{:}\AttributeTok{ }\StringTok{"response\_time"}
\AttributeTok{  }\FunctionTok{threshold}\KeywordTok{:}\AttributeTok{ }\StringTok{"\textless{} 200 ms"}
\AttributeTok{  }\FunctionTok{required\_success\_rate}\KeywordTok{:}\AttributeTok{ }\StringTok{"95\%"}
\AttributeTok{  }\FunctionTok{sample\_size}\KeywordTok{:}\AttributeTok{ }\DecValTok{20}
\end{Highlighting}
\end{Shaded}

\textbf{Runtime must handle uncertainty}: - Redundant components for
critical functions - Graceful degradation strategies - Continuous
monitoring with anomaly detection

\hypertarget{discrete-vs.-continuous}{%
\subsection{6.2 Discrete vs.~Continuous}\label{discrete-vs.-continuous}}

\hypertarget{the-difference-1}{%
\subsubsection{The Difference}\label{the-difference-1}}

\begin{longtable}[]{@{}ll@{}}
\toprule\noalign{}
Software & Biology \\
\midrule\noalign{}
\endhead
\bottomrule\noalign{}
\endlastfoot
Digital signals (0/1) & Analog signals (continuous) \\
Clear state boundaries & Gradual transitions \\
Instantaneous state changes & Progressive changes over time \\
Precise timing & Approximate timing \\
\end{longtable}

Software state transitions are instantaneous: a variable is either
\texttt{true} or \texttt{false}. Biological state transitions are
gradual: a muscle doesn't switch from ``relaxed'' to ``contracted'' but
transitions through a continuum.

\hypertarget{implications-for-wetware-engineering-1}{%
\subsubsection{Implications for Wetware
Engineering}\label{implications-for-wetware-engineering-1}}

\textbf{Interface parameters need tolerance ranges}:

\begin{Shaded}
\begin{Highlighting}[]
\FunctionTok{input}\KeywordTok{:}
\AttributeTok{  }\FunctionTok{voltage}\KeywordTok{:}
\AttributeTok{    }\FunctionTok{target}\KeywordTok{:}\AttributeTok{ 2.0 V}
\AttributeTok{    }\FunctionTok{tolerance}\KeywordTok{:}\AttributeTok{ ±0.2 V}
\AttributeTok{    }\FunctionTok{settling\_time}\KeywordTok{:}\AttributeTok{ 10 ms}
\end{Highlighting}
\end{Shaded}

\textbf{State definitions need thresholds}:

\begin{Shaded}
\begin{Highlighting}[]
\FunctionTok{states}\KeywordTok{:}
\AttributeTok{  }\FunctionTok{relaxed}\KeywordTok{:}\AttributeTok{ }\StringTok{"force \textless{} 5\% max"}
\AttributeTok{  }\FunctionTok{contracting}\KeywordTok{:}\AttributeTok{ }\StringTok{"5\% \textless{}= force \textless{} 95\% max"}
\AttributeTok{  }\FunctionTok{contracted}\KeywordTok{:}\AttributeTok{ }\StringTok{"force \textgreater{}= 95\% max"}
\AttributeTok{  }\FunctionTok{transition\_time}\KeywordTok{:}\AttributeTok{ }\StringTok{"50{-}200 ms"}
\end{Highlighting}
\end{Shaded}

\textbf{Timing specifications need ranges}:

\begin{Shaded}
\begin{Highlighting}[]
\FunctionTok{timing}\KeywordTok{:}
\AttributeTok{  }\FunctionTok{response\_time}\KeywordTok{:}\AttributeTok{ }\KeywordTok{\{}\AttributeTok{ }\FunctionTok{min}\KeywordTok{:}\AttributeTok{ }\DecValTok{100}\KeywordTok{,}\AttributeTok{ }\FunctionTok{typical}\KeywordTok{:}\AttributeTok{ }\DecValTok{150}\KeywordTok{,}\AttributeTok{ }\FunctionTok{max}\KeywordTok{:}\AttributeTok{ }\DecValTok{300}\KeywordTok{,}\AttributeTok{ }\FunctionTok{unit}\KeywordTok{:}\AttributeTok{ }\StringTok{"ms"}\AttributeTok{ }\KeywordTok{\}}
\AttributeTok{  }\FunctionTok{settling\_time}\KeywordTok{:}\AttributeTok{ }\KeywordTok{\{}\AttributeTok{ }\FunctionTok{typical}\KeywordTok{:}\AttributeTok{ }\DecValTok{500}\KeywordTok{,}\AttributeTok{ }\FunctionTok{max}\KeywordTok{:}\AttributeTok{ }\DecValTok{1000}\KeywordTok{,}\AttributeTok{ }\FunctionTok{unit}\KeywordTok{:}\AttributeTok{ }\StringTok{"ms"}\AttributeTok{ }\KeywordTok{\}}
\end{Highlighting}
\end{Shaded}

\hypertarget{isolation-vs.-coupling}{%
\subsection{6.3 Isolation vs.~Coupling}\label{isolation-vs.-coupling}}

\hypertarget{the-difference-2}{%
\subsubsection{The Difference}\label{the-difference-2}}

\begin{longtable}[]{@{}ll@{}}
\toprule\noalign{}
Software & Biology \\
\midrule\noalign{}
\endhead
\bottomrule\noalign{}
\endlastfoot
Process isolation (memory protection) & Shared chemical environment \\
No side effects (pure functions) & Systemic metabolic effects \\
Independent scaling & Resource competition \\
Clean interfaces & Signal crosstalk \\
\end{longtable}

Software processes are isolated by the operating system. A bug in one
process cannot corrupt another's memory. Biological components share
culture medium, and one component's metabolic waste affects all others.

\hypertarget{implications-for-wetware-engineering-2}{%
\subsubsection{Implications for Wetware
Engineering}\label{implications-for-wetware-engineering-2}}

\textbf{Explicit coupling declarations}:

\begin{Shaded}
\begin{Highlighting}[]
\FunctionTok{coupling}\KeywordTok{:}
\AttributeTok{  }\FunctionTok{metabolic}\KeywordTok{:}
\AttributeTok{    }\KeywordTok{{-}}\AttributeTok{ }\FunctionTok{component}\KeywordTok{:}\AttributeTok{ }\StringTok{"flexor"}
\AttributeTok{      }\FunctionTok{shares\_medium\_with}\KeywordTok{:}\AttributeTok{ }\KeywordTok{[}\StringTok{"extensor"}\KeywordTok{,}\AttributeTok{ }\StringTok{"sensor"}\KeywordTok{]}
\AttributeTok{      }\FunctionTok{waste\_products}\KeywordTok{:}\AttributeTok{ }\KeywordTok{[}\StringTok{"lactate"}\KeywordTok{,}\AttributeTok{ }\StringTok{"CO2"}\KeywordTok{]}
\AttributeTok{      }
\AttributeTok{  }\FunctionTok{electrical}\KeywordTok{:}
\AttributeTok{    }\KeywordTok{{-}}\AttributeTok{ }\FunctionTok{component}\KeywordTok{:}\AttributeTok{ }\StringTok{"controller"}
\AttributeTok{      }\FunctionTok{field\_effects\_on}\KeywordTok{:}\AttributeTok{ }\KeywordTok{[}\StringTok{"sensor"}\KeywordTok{]}
\AttributeTok{      }\FunctionTok{isolation\_required}\KeywordTok{:}\AttributeTok{ }\CharTok{true}
\end{Highlighting}
\end{Shaded}

\textbf{Isolation adapter specifications}:

\begin{Shaded}
\begin{Highlighting}[]
\FunctionTok{adapter}\KeywordTok{:}
\AttributeTok{  }\FunctionTok{type}\KeywordTok{:}\AttributeTok{ }\StringTok{"metabolic\_isolation"}
\AttributeTok{  }\FunctionTok{mechanism}\KeywordTok{:}\AttributeTok{ }\StringTok{"semipermeable\_membrane"}
\AttributeTok{  }\FunctionTok{allows}\KeywordTok{:}\AttributeTok{ }\KeywordTok{[}\StringTok{"glucose"}\KeywordTok{,}\AttributeTok{ }\StringTok{"oxygen"}\KeywordTok{,}\AttributeTok{ }\StringTok{"amino\_acids"}\KeywordTok{]}
\AttributeTok{  }\FunctionTok{blocks}\KeywordTok{:}\AttributeTok{ }\KeywordTok{[}\StringTok{"lactate \textgreater{} 10mM"}\KeywordTok{,}\AttributeTok{ }\StringTok{"inflammatory\_cytokines"}\KeywordTok{]}
\end{Highlighting}
\end{Shaded}

\textbf{System-level resource budgeting}:

\begin{Shaded}
\begin{Highlighting}[]
\FunctionTok{resource\_budget}\KeywordTok{:}
\AttributeTok{  }\FunctionTok{oxygen}\KeywordTok{:}
\AttributeTok{    }\FunctionTok{supply}\KeywordTok{:}\AttributeTok{ 100 µmol/hour}
\AttributeTok{    }\FunctionTok{consumers}\KeywordTok{:}
\AttributeTok{      }\KeywordTok{{-}}\AttributeTok{ }\FunctionTok{flexor}\KeywordTok{:}\AttributeTok{ 30 µmol/hour}
\AttributeTok{      }\KeywordTok{{-}}\AttributeTok{ }\FunctionTok{extensor}\KeywordTok{:}\AttributeTok{ 30 µmol/hour}
\AttributeTok{      }\KeywordTok{{-}}\AttributeTok{ }\FunctionTok{controller}\KeywordTok{:}\AttributeTok{ 20 µmol/hour}
\AttributeTok{      }\KeywordTok{{-}}\AttributeTok{ }\FunctionTok{sensor}\KeywordTok{:}\AttributeTok{ 5 µmol/hour}
\AttributeTok{    }\FunctionTok{margin}\KeywordTok{:}\AttributeTok{ 15\%}
\end{Highlighting}
\end{Shaded}

\hypertarget{the-immune-system-no-software-equivalent}{%
\subsection{6.4 The Immune System: No Software
Equivalent}\label{the-immune-system-no-software-equivalent}}

\hypertarget{the-challenge}{%
\subsubsection{The Challenge}\label{the-challenge}}

Software components do not ``reject'' each other. Biological components
from different genetic backgrounds may trigger immune responses ranging
from mild inflammation to complete destruction.

This has no software parallel. The closest analogy might be software
license incompatibility, but that's a legal/social construct, not a
physical phenomenon.

\hypertarget{required-innovations}{%
\subsubsection{Required Innovations}\label{required-innovations}}

\textbf{Immune compatibility scoring}:

\begin{Shaded}
\begin{Highlighting}[]
\FunctionTok{immune\_profile}\KeywordTok{:}
\AttributeTok{  }\FunctionTok{source}\KeywordTok{:}\AttributeTok{ }\StringTok{"donor\_42"}
\AttributeTok{  }\FunctionTok{hla\_type}\KeywordTok{:}
\AttributeTok{    }\FunctionTok{class\_i}\KeywordTok{:}\AttributeTok{ }\KeywordTok{[}\StringTok{"A*02:01"}\KeywordTok{,}\AttributeTok{ }\StringTok{"B*07:02"}\KeywordTok{,}\AttributeTok{ }\StringTok{"C*07:01"}\KeywordTok{]}
\AttributeTok{    }\FunctionTok{class\_ii}\KeywordTok{:}\AttributeTok{ }\KeywordTok{[}\StringTok{"DRB1*04:01"}\KeywordTok{,}\AttributeTok{ }\StringTok{"DQB1*03:02"}\KeywordTok{]}
\AttributeTok{  }
\AttributeTok{  }\FunctionTok{compatibility}\KeywordTok{:}
\AttributeTok{    }\FunctionTok{autologous}\KeywordTok{:}\AttributeTok{ }\FloatTok{1.0}\CommentTok{      \# Same donor: perfect}
\AttributeTok{    }\FunctionTok{hla\_matched}\KeywordTok{:}\AttributeTok{ }\FloatTok{0.85}\CommentTok{    \# Matched donor: good}
\AttributeTok{    }\FunctionTok{allogeneic}\KeywordTok{:}\AttributeTok{ }\FloatTok{0.3}\CommentTok{      \# Random donor: risky}
\AttributeTok{    }\FunctionTok{xenogeneic}\KeywordTok{:}\AttributeTok{ }\FloatTok{0.05}\CommentTok{     \# Different species: very risky}
\end{Highlighting}
\end{Shaded}

\textbf{Isolation barrier specifications}:

\begin{Shaded}
\begin{Highlighting}[]
\FunctionTok{immune\_barrier}\KeywordTok{:}
\AttributeTok{  }\FunctionTok{type}\KeywordTok{:}\AttributeTok{ }\StringTok{"encapsulation"}
\AttributeTok{  }\FunctionTok{material}\KeywordTok{:}\AttributeTok{ }\StringTok{"alginate\_hydrogel"}
\AttributeTok{  }\FunctionTok{pore\_size}\KeywordTok{:}\AttributeTok{ }\StringTok{"100 kDa MWCO"}
\AttributeTok{  }
\AttributeTok{  }\FunctionTok{permeability}\KeywordTok{:}
\AttributeTok{    }\FunctionTok{oxygen}\KeywordTok{:}\AttributeTok{ }\StringTok{"high"}
\AttributeTok{    }\FunctionTok{glucose}\KeywordTok{:}\AttributeTok{ }\StringTok{"high"}
\AttributeTok{    }\FunctionTok{insulin}\KeywordTok{:}\AttributeTok{ }\StringTok{"high"}
\AttributeTok{    }\FunctionTok{antibodies}\KeywordTok{:}\AttributeTok{ }\StringTok{"blocked"}
\AttributeTok{    }\FunctionTok{immune\_cells}\KeywordTok{:}\AttributeTok{ }\StringTok{"blocked"}
\AttributeTok{  }
\AttributeTok{  }\FunctionTok{expected\_lifetime}\KeywordTok{:}\AttributeTok{ }\StringTok{"6 months"}
\AttributeTok{  }\FunctionTok{failure\_mode}\KeywordTok{:}\AttributeTok{ }\StringTok{"fibrotic\_overgrowth"}
\end{Highlighting}
\end{Shaded}

\textbf{Compatibility checking in Bio-DSL}:

\begin{Shaded}
\begin{Highlighting}[]
\NormalTok{COMPONENT heart FROM "cardiomyocyte{-}human@2.0" \{}
\NormalTok{  donor: "donor\_42"}
\NormalTok{\}}

\NormalTok{COMPONENT vessel FROM "endothelial{-}human@1.5" \{}
\NormalTok{  donor: "donor\_42"  // Same donor: compatible}
\NormalTok{\}}

\NormalTok{// Compiler warning if donors don\textquotesingle{}t match:}
\NormalTok{// WARNING: Immune compatibility not verified between }
\NormalTok{//          heart (donor\_42) and vessel (donor\_17)}
\NormalTok{//          Consider: HLA matching or isolation barrier}
\end{Highlighting}
\end{Shaded}

\hypertarget{living-degradation-1}{%
\subsection{6.5 Living Degradation}\label{living-degradation-1}}

\hypertarget{the-challenge-1}{%
\subsubsection{The Challenge}\label{the-challenge-1}}

Software does not age. A function written in 1990 executes identically
today (given compatible runtime). Biological components inherently
degrade: cells senesce, proteins denature, structures weaken.

\hypertarget{required-innovations-1}{%
\subsubsection{Required Innovations}\label{required-innovations-1}}

\textbf{Degradation modeling}:

\begin{Shaded}
\begin{Highlighting}[]
\FunctionTok{degradation}\KeywordTok{:}
\AttributeTok{  }\FunctionTok{model}\KeywordTok{:}\AttributeTok{ }\StringTok{"weibull"}
\AttributeTok{  }\FunctionTok{parameters}\KeywordTok{:}
\AttributeTok{    }\FunctionTok{shape}\KeywordTok{:}\AttributeTok{ }\FloatTok{2.5}
\AttributeTok{    }\FunctionTok{scale}\KeywordTok{:}\AttributeTok{ }\DecValTok{14}\CommentTok{  \# days}
\AttributeTok{  }
\AttributeTok{  }\FunctionTok{markers}\KeywordTok{:}
\AttributeTok{    }\FunctionTok{early\_warning}\KeywordTok{:}
\AttributeTok{      }\KeywordTok{{-}}\AttributeTok{ }\StringTok{"force\_output \textless{} 90\% baseline"}
\AttributeTok{      }\KeywordTok{{-}}\AttributeTok{ }\StringTok{"response\_time \textgreater{} 120\% baseline"}
\AttributeTok{    }
\AttributeTok{    }\FunctionTok{end\_of\_life}\KeywordTok{:}
\AttributeTok{      }\KeywordTok{{-}}\AttributeTok{ }\StringTok{"viability \textless{} 70\%"}
\AttributeTok{      }\KeywordTok{{-}}\AttributeTok{ }\StringTok{"force\_output \textless{} 50\% baseline"}
\end{Highlighting}
\end{Shaded}

\textbf{Maintenance protocols}:

\begin{Shaded}
\begin{Highlighting}[]
\FunctionTok{maintenance}\KeywordTok{:}
\AttributeTok{  }\FunctionTok{routine}\KeywordTok{:}
\AttributeTok{    }\KeywordTok{{-}}\AttributeTok{ }\FunctionTok{action}\KeywordTok{:}\AttributeTok{ }\StringTok{"medium\_change"}
\AttributeTok{      }\FunctionTok{frequency}\KeywordTok{:}\AttributeTok{ }\StringTok{"every 48 hours"}
\AttributeTok{    }\KeywordTok{{-}}\AttributeTok{ }\FunctionTok{action}\KeywordTok{:}\AttributeTok{ }\StringTok{"viability\_check"}
\AttributeTok{      }\FunctionTok{frequency}\KeywordTok{:}\AttributeTok{ }\StringTok{"daily"}
\AttributeTok{  }
\AttributeTok{  }\FunctionTok{corrective}\KeywordTok{:}
\AttributeTok{    }\KeywordTok{{-}}\AttributeTok{ }\FunctionTok{trigger}\KeywordTok{:}\AttributeTok{ }\StringTok{"force\_decline \textgreater{} 20\%"}
\AttributeTok{      }\FunctionTok{action}\KeywordTok{:}\AttributeTok{ }\StringTok{"increase\_growth\_factors"}
\AttributeTok{    }\KeywordTok{{-}}\AttributeTok{ }\FunctionTok{trigger}\KeywordTok{:}\AttributeTok{ }\StringTok{"viability \textless{} 80\%"}
\AttributeTok{      }\FunctionTok{action}\KeywordTok{:}\AttributeTok{ }\StringTok{"partial\_medium\_refresh"}
\end{Highlighting}
\end{Shaded}

\textbf{Replacement strategies}:

\begin{Shaded}
\begin{Highlighting}[]
\FunctionTok{replacement}\KeywordTok{:}
\AttributeTok{  }\FunctionTok{strategy}\KeywordTok{:}\AttributeTok{ }\StringTok{"hot\_swap"}
\AttributeTok{  }\FunctionTok{trigger}\KeywordTok{:}\AttributeTok{ }\StringTok{"viability \textless{} 70\%"}
\AttributeTok{  }
\AttributeTok{  }\FunctionTok{procedure}\KeywordTok{:}
\AttributeTok{    }\FunctionTok{1}\KeywordTok{:}\AttributeTok{ }\StringTok{"activate\_backup\_component"}
\AttributeTok{    }\FunctionTok{2}\KeywordTok{:}\AttributeTok{ }\StringTok{"transfer\_connections"}
\AttributeTok{    }\FunctionTok{3}\KeywordTok{:}\AttributeTok{ }\StringTok{"verify\_function"}
\AttributeTok{    }\FunctionTok{4}\KeywordTok{:}\AttributeTok{ }\StringTok{"remove\_degraded\_component"}
\AttributeTok{  }
\AttributeTok{  }\FunctionTok{backup\_inventory}\KeywordTok{:}\AttributeTok{ }\DecValTok{2}\CommentTok{  \# Keep 2 spares ready}
\end{Highlighting}
\end{Shaded}

\hypertarget{ethical-constraints-1}{%
\subsection{6.6 Ethical Constraints}\label{ethical-constraints-1}}

\hypertarget{the-challenge-2}{%
\subsubsection{The Challenge}\label{the-challenge-2}}

Software has no inherent ethical status. Biological components,
especially those involving human cells or neural tissue, raise ethical
considerations:

\begin{itemize}
\tightlist
\item
  \textbf{Source ethics}: How were cells obtained? Was there informed
  consent?
\item
  \textbf{Capability ethics}: Could the assembly develop consciousness
  or sentience?
\item
  \textbf{Use ethics}: What applications are acceptable?
\item
  \textbf{Disposal ethics}: How should biological materials be
  destroyed?
\end{itemize}

\hypertarget{required-innovations-2}{%
\subsubsection{Required Innovations}\label{required-innovations-2}}

\textbf{Ethical metadata}:

\begin{Shaded}
\begin{Highlighting}[]
\FunctionTok{ethics}\KeywordTok{:}
\AttributeTok{  }\FunctionTok{source}\KeywordTok{:}
\AttributeTok{    }\FunctionTok{consent\_type}\KeywordTok{:}\AttributeTok{ }\StringTok{"informed\_written"}
\AttributeTok{    }\FunctionTok{consent\_scope}\KeywordTok{:}\AttributeTok{ }\StringTok{"research\_only"}
\AttributeTok{    }\FunctionTok{donor\_compensation}\KeywordTok{:}\AttributeTok{ }\StringTok{"none"}
\AttributeTok{    }\FunctionTok{irb\_approval}\KeywordTok{:}\AttributeTok{ }\StringTok{"IRB{-}2025{-}0142"}
\AttributeTok{  }
\AttributeTok{  }\FunctionTok{constraints}\KeywordTok{:}
\AttributeTok{    }\FunctionTok{prohibited\_uses}\KeywordTok{:}
\AttributeTok{      }\KeywordTok{{-}}\AttributeTok{ }\StringTok{"reproductive\_cloning"}
\AttributeTok{      }\KeywordTok{{-}}\AttributeTok{ }\StringTok{"consciousness\_creation"}
\AttributeTok{      }\KeywordTok{{-}}\AttributeTok{ }\StringTok{"military\_applications"}
\AttributeTok{    }
\AttributeTok{    }\FunctionTok{required\_oversight}\KeywordTok{:}
\AttributeTok{      }\KeywordTok{{-}}\AttributeTok{ }\StringTok{"irb\_review\_annual"}
\AttributeTok{      }\KeywordTok{{-}}\AttributeTok{ }\StringTok{"ethics\_board\_notification"}
\AttributeTok{  }
\AttributeTok{  }\FunctionTok{disposal}\KeywordTok{:}
\AttributeTok{    }\FunctionTok{method}\KeywordTok{:}\AttributeTok{ }\StringTok{"autoclave\_and\_biohazard\_waste"}
\AttributeTok{    }\FunctionTok{documentation}\KeywordTok{:}\AttributeTok{ }\StringTok{"required"}
\end{Highlighting}
\end{Shaded}

\textbf{Capability limits in Bio-DSL}:

\begin{Shaded}
\begin{Highlighting}[]
\NormalTok{// Compiler enforces ethical constraints}
\NormalTok{SYSTEM brain\_organoid\_array \{}
\NormalTok{  // ERROR: Assembly exceeds neural complexity threshold}
\NormalTok{  // Maximum allowed: 10\^{}6 neurons}
\NormalTok{  // This assembly: 10\^{}8 neurons}
\NormalTok{  // Requires: Enhanced ethics review}
  
\NormalTok{  COMPONENT organoid[100] FROM "brain{-}organoid@1.0"}
\NormalTok{  CONNECT organoid[*] IN mesh\_topology}
\NormalTok{\}}
\end{Highlighting}
\end{Shaded}

\hypertarget{summary-the-innovation-agenda}{%
\subsection{6.7 Summary: The Innovation
Agenda}\label{summary-the-innovation-agenda}}

\begin{longtable}[]{@{}
  >{\raggedright\arraybackslash}p{(\columnwidth - 4\tabcolsep) * \real{0.2157}}
  >{\raggedright\arraybackslash}p{(\columnwidth - 4\tabcolsep) * \real{0.3922}}
  >{\raggedright\arraybackslash}p{(\columnwidth - 4\tabcolsep) * \real{0.3922}}@{}}
\toprule\noalign{}
\begin{minipage}[b]{\linewidth}\raggedright
Challenge
\end{minipage} & \begin{minipage}[b]{\linewidth}\raggedright
Software Equivalent
\end{minipage} & \begin{minipage}[b]{\linewidth}\raggedright
Required Innovation
\end{minipage} \\
\midrule\noalign{}
\endhead
\bottomrule\noalign{}
\endlastfoot
Stochasticity & None (deterministic) & Probabilistic contracts,
statistical testing \\
Continuous states & None (discrete) & Tolerance ranges, threshold
definitions \\
Metabolic coupling & None (isolated) & Coupling declarations, resource
budgeting \\
Immune rejection & None & Compatibility scoring, isolation barriers \\
Living degradation & None & Degradation models, maintenance protocols \\
Ethical constraints & Licensing (weak analogy) & Ethical metadata,
capability limits \\
\end{longtable}

These challenges do not invalidate the paradigm transfer---they define
the research agenda for making it complete. \# 7. Related Work and
Positioning

\hypertarget{synthetic-biology-and-standardization}{%
\subsection{7.1 Synthetic Biology and
Standardization}\label{synthetic-biology-and-standardization}}

\hypertarget{biobricks-and-igem}{%
\subsubsection{BioBricks and iGEM}\label{biobricks-and-igem}}

The BioBricks Foundation and iGEM (International Genetically Engineered
Machine) competition pioneered biological standardization at the genetic
level. BioBricks defined standard assembly methods for DNA parts,
enabling students worldwide to combine genetic components.

\textbf{Relationship to Wetware Engineering}: - BioBricks operates at
the \textbf{genetic level} (DNA sequences) - Wetware Engineering
operates at the \textbf{organ/system level} (tissues, organoids) - They
are \textbf{complementary}: BioBricks could define the genetic content
\emph{within} a Bio-Component

\hypertarget{sbol-synthetic-biology-open-language}{%
\subsubsection{SBOL (Synthetic Biology Open
Language)}\label{sbol-synthetic-biology-open-language}}

SBOL provides a standardized data format for describing genetic designs,
enabling exchange between software tools and laboratories.

\textbf{Relationship to Wetware Engineering}: - SBOL describes
\textbf{what genes are in a component} - Bio-Component Spec describes
\textbf{what the component does as a functional unit} - SBOL could be
embedded within Bio-Component Spec for genetic traceability

\begin{Shaded}
\begin{Highlighting}[]
\CommentTok{\# Bio{-}Component with embedded SBOL reference}
\FunctionTok{source}\KeywordTok{:}
\AttributeTok{  }\FunctionTok{genetic\_design}\KeywordTok{:}
\AttributeTok{    }\FunctionTok{sbol\_uri}\KeywordTok{:}\AttributeTok{ }\StringTok{"https://synbiohub.org/design/muscle{-}v2"}
\AttributeTok{    }\FunctionTok{modifications}\KeywordTok{:}\AttributeTok{ }\KeywordTok{[}\StringTok{"GFP reporter"}\KeywordTok{,}\AttributeTok{ }\StringTok{"tetracycline{-}inducible"}\KeywordTok{]}
\end{Highlighting}
\end{Shaded}

\hypertarget{comparison-table}{%
\subsubsection{Comparison Table}\label{comparison-table}}

\begin{longtable}[]{@{}
  >{\raggedright\arraybackslash}p{(\columnwidth - 4\tabcolsep) * \real{0.1818}}
  >{\raggedright\arraybackslash}p{(\columnwidth - 4\tabcolsep) * \real{0.3409}}
  >{\raggedright\arraybackslash}p{(\columnwidth - 4\tabcolsep) * \real{0.4773}}@{}}
\toprule\noalign{}
\begin{minipage}[b]{\linewidth}\raggedright
Aspect
\end{minipage} & \begin{minipage}[b]{\linewidth}\raggedright
BioBricks/SBOL
\end{minipage} & \begin{minipage}[b]{\linewidth}\raggedright
Wetware Engineering
\end{minipage} \\
\midrule\noalign{}
\endhead
\bottomrule\noalign{}
\endlastfoot
Abstraction level & Genetic (DNA) & Organ/System (tissue) \\
Unit of composition & DNA part & Functional module \\
Assembly method & Restriction enzymes, Gibson & Physical/fluidic
connection \\
Standardization focus & Sequence format & Interface protocol \\
Primary users & Molecular biologists & Tissue engineers, roboticists \\
\end{longtable}

\hypertarget{organ-on-chip-and-organoids}{%
\subsection{7.2 Organ-on-Chip and
Organoids}\label{organ-on-chip-and-organoids}}

\hypertarget{organ-on-chip-technology}{%
\subsubsection{Organ-on-Chip
Technology}\label{organ-on-chip-technology}}

Organ-on-chip devices culture human cells in microfluidic environments
that mimic organ physiology. Companies like Emulate and TissUse have
commercialized multi-organ systems.

\textbf{Relationship to Wetware Engineering}: - Organ-on-chip provides
\textbf{physical implementations} of Bio-Components - Current systems
lack \textbf{standardized interfaces} between chips - Wetware
Engineering provides the \textbf{abstraction framework} they need

\hypertarget{organoid-research}{%
\subsubsection{Organoid Research}\label{organoid-research}}

Organoids are self-organizing 3D tissue cultures that recapitulate organ
structure and function. Brain organoids, gut organoids, and kidney
organoids have advanced rapidly.

\textbf{Relationship to Wetware Engineering}: - Organoids are excellent
\textbf{candidate Bio-Components} - Current organoid research focuses on
\textbf{individual organs}, not composition - Wetware Engineering
provides \textbf{composition methodology}

\hypertarget{whats-missing}{%
\subsubsection{What's Missing}\label{whats-missing}}

\begin{longtable}[]{@{}
  >{\raggedright\arraybackslash}p{(\columnwidth - 2\tabcolsep) * \real{0.3590}}
  >{\raggedright\arraybackslash}p{(\columnwidth - 2\tabcolsep) * \real{0.6410}}@{}}
\toprule\noalign{}
\begin{minipage}[b]{\linewidth}\raggedright
Current State
\end{minipage} & \begin{minipage}[b]{\linewidth}\raggedright
Wetware Engineering Adds
\end{minipage} \\
\midrule\noalign{}
\endhead
\bottomrule\noalign{}
\endlastfoot
Each lab develops proprietary protocols & Standardized specifications \\
No interface standards between organs & Bio-Interface Protocol \\
Results described in papers & Machine-readable descriptions \\
Composition is ad-hoc & Declarative composition language \\
No version control & Semantic versioning \\
\end{longtable}

\hypertarget{systems-biology-modeling}{%
\subsection{7.3 Systems Biology
Modeling}\label{systems-biology-modeling}}

\hypertarget{sbml-systems-biology-markup-language}{%
\subsubsection{SBML (Systems Biology Markup
Language)}\label{sbml-systems-biology-markup-language}}

SBML is a standard format for representing computational models of
biological processes, particularly biochemical reaction networks.

\hypertarget{cellml}{%
\subsubsection{CellML}\label{cellml}}

CellML describes mathematical models of cellular function, enabling
model sharing and reuse.

\textbf{Relationship to Wetware Engineering}: - SBML/CellML describe
\textbf{how components behave internally} (simulation) - Bio-DSL
describes \textbf{how components connect externally} (composition) -
They serve different purposes and can be used together

\begin{Shaded}
\begin{Highlighting}[]
\CommentTok{\# Bio{-}Component with SBML behavior model}
\FunctionTok{behavior}\KeywordTok{:}
\AttributeTok{  }\FunctionTok{model\_type}\KeywordTok{:}\AttributeTok{ }\StringTok{"SBML"}
\AttributeTok{  }\FunctionTok{model\_uri}\KeywordTok{:}\AttributeTok{ }\StringTok{"https://biomodels.org/MODEL123"}
\AttributeTok{  }\FunctionTok{parameters}\KeywordTok{:}
\AttributeTok{    }\KeywordTok{{-}}\AttributeTok{ }\FunctionTok{name}\KeywordTok{:}\AttributeTok{ }\StringTok{"contraction\_rate"}
\AttributeTok{      }\FunctionTok{value}\KeywordTok{:}\AttributeTok{ }\FloatTok{0.5}
\AttributeTok{      }\FunctionTok{unit}\KeywordTok{:}\AttributeTok{ }\StringTok{"1/s"}
\end{Highlighting}
\end{Shaded}

\hypertarget{comparison}{%
\subsubsection{Comparison}\label{comparison}}

\begin{longtable}[]{@{}lll@{}}
\toprule\noalign{}
Aspect & SBML/CellML & Bio-DSL \\
\midrule\noalign{}
\endhead
\bottomrule\noalign{}
\endlastfoot
Purpose & Simulate behavior & Describe composition \\
Focus & Internal dynamics & External connections \\
Output & Simulation results & Runtime configuration \\
Users & Computational biologists & System builders \\
\end{longtable}

\hypertarget{biohybrid-robotics}{%
\subsection{7.4 Biohybrid Robotics}\label{biohybrid-robotics}}

\hypertarget{living-machines}{%
\subsubsection{Living Machines}\label{living-machines}}

Research groups have created robots powered by biological actuators:
muscle-powered swimmers, insect-machine hybrids, and biohybrid grippers.

\textbf{Relationship to Wetware Engineering}: - Biohybrid robotics
demonstrates \textbf{feasibility} of biological components - Current
work is \textbf{bespoke}---each system designed from scratch - Wetware
Engineering provides \textbf{reusable framework}

\hypertarget{key-publications}{%
\subsubsection{Key Publications}\label{key-publications}}

\begin{itemize}
\tightlist
\item
  Raman \& Bashir (2017): Biohybrid actuators review
\item
  Ricotti et al.~(2017): Biohybrid systems for robotics
\item
  Park et al.~(2016): Muscle-powered swimming robot
\end{itemize}

These works prove biological components can be engineered. Wetware
Engineering asks: how do we make this \textbf{systematic and
reproducible}?

\hypertarget{software-engineering-for-biology}{%
\subsection{7.5 Software Engineering for
Biology}\label{software-engineering-for-biology}}

\hypertarget{laboratory-automation}{%
\subsubsection{Laboratory Automation}\label{laboratory-automation}}

Tools like Autoprotocol and Antha provide programming languages for
laboratory procedures, enabling reproducible experiments.

\textbf{Relationship to Wetware Engineering}: - Lab automation describes
\textbf{how to make components} - Wetware Engineering describes
\textbf{how to compose components} - They address different phases of
the development lifecycle

\hypertarget{workflow-systems}{%
\subsubsection{Workflow Systems}\label{workflow-systems}}

Galaxy, Nextflow, and Snakemake manage computational biology workflows.

\textbf{Relationship to Wetware Engineering}: - Workflow systems manage
\textbf{data analysis pipelines} - Bio-DSL manages \textbf{physical
system composition} - Different domains, similar abstraction principles

\hypertarget{positioning-summary}{%
\subsection{7.6 Positioning Summary}\label{positioning-summary}}

\begin{verbatim}
                    Abstraction Level
                    
    High    ┌─────────────────────────────────┐
            │     Wetware Engineering          │
            │     (System Composition)         │
            └─────────────────────────────────┘
                           ↑
                    Uses components from
                           ↑
    Medium  ┌─────────────────────────────────┐
            │   Organ-on-Chip / Organoids      │
            │   (Physical Implementation)      │
            └─────────────────────────────────┘
                           ↑
                    Built using
                           ↑
    Low     ┌─────────────────────────────────┐
            │   BioBricks / SBOL / SBML        │
            │   (Genetic / Molecular)          │
            └─────────────────────────────────┘
\end{verbatim}

\textbf{Wetware Engineering's unique contribution}: Providing the
\textbf{system-level abstraction layer} that connects molecular/genetic
engineering to functional biological systems, using software engineering
principles.

\hypertarget{what-we-are-not-claiming}{%
\subsection{7.7 What We Are NOT
Claiming}\label{what-we-are-not-claiming}}

To be clear about scope:

\begin{enumerate}
\def\labelenumi{\arabic{enumi}.}
\item
  \textbf{We are not claiming to have built working systems}. This paper
  proposes a framework; implementation is future work.
\item
  \textbf{We are not claiming biology is ``just like'' software}.
  Section 6 details fundamental differences requiring novel solutions.
\item
  \textbf{We are not claiming to replace existing approaches}. We
  complement synthetic biology, organoid research, and systems biology.
\item
  \textbf{We are not claiming immediate practical application}. The
  roadmap spans decades.
\end{enumerate}

What we ARE claiming: \textbf{The conceptual framework of software
engineering---modularity, interfaces, composition, versioning---provides
valuable abstractions for biological system construction, and
articulating this framework is a necessary first step.} \# 8. Conclusion
and Future Directions

\hypertarget{summary-of-contributions}{%
\subsection{8.1 Summary of
Contributions}\label{summary-of-contributions}}

This paper has proposed \textbf{Wetware Engineering}, a
cross-disciplinary methodology that systematically transfers software
engineering paradigms to biological system construction. Our
contributions are:

\hypertarget{conceptual-framework}{%
\subsubsection{Conceptual Framework}\label{conceptual-framework}}

We defined the \textbf{Component-Interface-Runtime triad} as the
foundational abstraction for modular biological systems: -
\textbf{Bio-Component}: Self-contained functional biological units -
\textbf{Bio-Interface}: Standardized connection protocols across four
dimensions (power, signal, isolation, mechanical) -
\textbf{Bio-Runtime}: Orchestration layer for resource management,
monitoring, and fault handling

\hypertarget{technical-specifications}{%
\subsubsection{Technical
Specifications}\label{technical-specifications}}

We proposed concrete specifications: - \textbf{Bio-Component Spec v0.1}:
A standardized schema for describing biological modules, including
metadata, interfaces, requirements, performance metrics, and testing -
\textbf{Bio-DSL}: A domain-specific language for declarative system
composition, with constructs for component declaration, connection,
runtime configuration, and behavioral logic

\hypertarget{systematic-analysis}{%
\subsubsection{Systematic Analysis}\label{systematic-analysis}}

We provided systematic mappings between software and biological
engineering: - \textbf{Direct mappings}: Versioning, dependency
declaration, documentation - \textbf{Analogous mappings}: Testing, error
handling, logging - \textbf{Novel challenges}: Immune compatibility,
metabolic coupling, living degradation, ethical constraints

\hypertarget{honest-assessment}{%
\subsubsection{Honest Assessment}\label{honest-assessment}}

We identified fundamental differences that require innovation beyond
paradigm transfer, establishing a research agenda for the field.

\hypertarget{the-path-forward}{%
\subsection{8.2 The Path Forward}\label{the-path-forward}}

\hypertarget{phase-1-foundation-1-3-years}{%
\subsubsection{Phase 1: Foundation (1-3
years)}\label{phase-1-foundation-1-3-years}}

\textbf{Goals}: - Refine specifications based on community feedback -
Develop proof-of-concept tooling (parser, validator) - Document 10-20
existing biological systems using Bio-Component Spec - Publish reference
implementations

\textbf{Milestones}: - Bio-Component Spec v1.0 release - Bio-DSL parser
and validator - First community-contributed component specifications -
Workshop or symposium on wetware engineering

\hypertarget{phase-2-validation-3-7-years}{%
\subsubsection{Phase 2: Validation (3-7
years)}\label{phase-2-validation-3-7-years}}

\textbf{Goals}: - Physically implement 2-3 component systems using the
framework - Validate that standardized descriptions improve
reproducibility - Develop interface adapters for common connection types
- Build component registry infrastructure

\textbf{Milestones}: - First physically assembled system from Bio-DSL
specification - Reproducibility study: same spec, different labs -
Component registry with 100+ entries - Integration with existing tools
(SBOL, SBML)

\hypertarget{phase-3-ecosystem-7-15-years}{%
\subsubsection{Phase 3: Ecosystem (7-15
years)}\label{phase-3-ecosystem-7-15-years}}

\textbf{Goals}: - Establish community governance for standards -
Commercial adoption in drug discovery, tissue engineering - Educational
curriculum development - International standardization (ISO, IEEE)

\textbf{Milestones}: - Industry consortium formation - First commercial
products using Bio-Component standards - University courses on wetware
engineering - Formal standardization process initiated

\hypertarget{call-to-action}{%
\subsection{8.3 Call to Action}\label{call-to-action}}

Wetware Engineering requires contributions from multiple communities:

\hypertarget{for-biologists-and-tissue-engineers}{%
\subsubsection{For Biologists and Tissue
Engineers}\label{for-biologists-and-tissue-engineers}}

\begin{itemize}
\tightlist
\item
  \textbf{Describe your work} using Bio-Component Spec format
\item
  \textbf{Identify interface requirements} that would enable composition
\item
  \textbf{Share protocols} in machine-readable formats
\item
  \textbf{Provide feedback} on specification usability
\end{itemize}

\hypertarget{for-software-engineers}{%
\subsubsection{For Software Engineers}\label{for-software-engineers}}

\begin{itemize}
\tightlist
\item
  \textbf{Contribute tooling}: parsers, validators, editors
\item
  \textbf{Apply design patterns} to biological contexts
\item
  \textbf{Develop testing frameworks} adapted for biological variability
\item
  \textbf{Build infrastructure}: registries, package managers
\end{itemize}

\hypertarget{for-standards-bodies}{%
\subsubsection{For Standards Bodies}\label{for-standards-bodies}}

\begin{itemize}
\tightlist
\item
  \textbf{Engage early} in specification development
\item
  \textbf{Coordinate} with existing biological standards (SBOL, SBML)
\item
  \textbf{Consider} biological-specific requirements (ethics, safety)
\end{itemize}

\hypertarget{for-funding-agencies}{%
\subsubsection{For Funding Agencies}\label{for-funding-agencies}}

\begin{itemize}
\tightlist
\item
  \textbf{Support cross-disciplinary} methodology research
\item
  \textbf{Fund infrastructure} (registries, tools) not just applications
\item
  \textbf{Enable long-term} projects (this is a decades-long endeavor)
\end{itemize}

\hypertarget{limitations-and-caveats}{%
\subsection{8.4 Limitations and Caveats}\label{limitations-and-caveats}}

We acknowledge significant limitations:

\begin{enumerate}
\def\labelenumi{\arabic{enumi}.}
\item
  \textbf{No experimental validation}: This paper proposes a framework;
  we have not physically built systems using it.
\item
  \textbf{Specification incompleteness}: Bio-Component Spec v0.1 is a
  starting point, not a finished standard.
\item
  \textbf{Tooling absence}: The envisioned toolchain does not yet exist.
\item
  \textbf{Community adoption uncertainty}: Standards succeed through
  adoption, which cannot be guaranteed.
\item
  \textbf{Biological complexity}: Real biological systems may resist the
  clean abstractions we propose.
\end{enumerate}

These limitations do not invalidate the approach---they define the work
ahead.

\hypertarget{closing-thoughts}{%
\subsection{8.5 Closing Thoughts}\label{closing-thoughts}}

Software engineering transformed from craft to discipline over five
decades. The journey included: - Conceptual breakthroughs (structured
programming, object-orientation) - Standardization efforts (ASCII,
TCP/IP, HTTP) - Tool development (compilers, IDEs, version control) -
Community building (open source, Stack Overflow) - Educational
formalization (CS degrees, bootcamps)

Biological engineering stands at a similar inflection point. The
question is not whether modularization will come to biology---the
complexity of biological systems demands it---but how quickly and how
well.

We offer Wetware Engineering not as a finished solution, but as a
\textbf{conceptual framework} and \textbf{conversation starter}. The
goal is to accelerate the transition from artisanal biological
construction to systematic biological engineering.

\begin{quote}
``Software engineering took 50 years to evolve from monolithic
applications to microservices architecture. We hope biological
engineering won't need another 50 years.''
\end{quote}

The tools of software engineering---abstraction, modularity,
standardization, composition---are not specific to silicon. They are
\textbf{general principles of managing complexity}. Biology is complex.
These principles can help.

The future of biological engineering is modular. The question is: will
we design that future deliberately, or stumble into it accidentally?

We choose to design.

\begin{center}\rule{0.5\linewidth}{0.5pt}\end{center}

\hypertarget{acknowledgments}{%
\subsection{Acknowledgments}\label{acknowledgments}}

{[}To be added{]}

\hypertarget{author-contributions}{%
\subsection{Author Contributions}\label{author-contributions}}

{[}Author name{]} conceived the Wetware Engineering concept, designed
the technical specifications, and wrote the manuscript.

\hypertarget{competing-interests}{%
\subsection{Competing Interests}\label{competing-interests}}

The author declares no competing interests.

\hypertarget{data-availability}{%
\subsection{Data Availability}\label{data-availability}}

All specifications and examples are available at:
https://github.com/tukuaiai/wetware-engineering

\hypertarget{references}{%
\subsection{References}\label{references}}

{[}See separate references section{]} \# References

\hypertarget{software-engineering-foundations}{%
\subsection{Software Engineering
Foundations}\label{software-engineering-foundations}}

\begin{enumerate}
\def\labelenumi{\arabic{enumi}.}
\item
  Parnas, D. L. (1972). On the criteria to be used in decomposing
  systems into modules. \emph{Communications of the ACM}, 15(12),
  1053-1058.
\item
  Dijkstra, E. W. (1968). Go to statement considered harmful.
  \emph{Communications of the ACM}, 11(3), 147-148.
\item
  Gamma, E., Helm, R., Johnson, R., \& Vlissides, J. (1994).
  \emph{Design Patterns: Elements of Reusable Object-Oriented Software}.
  Addison-Wesley.
\item
  Martin, R. C. (2003). \emph{Agile Software Development: Principles,
  Patterns, and Practices}. Prentice Hall.
\item
  Fowler, M. (2010). \emph{Domain-Specific Languages}. Addison-Wesley.
\item
  Brooks, F. P. (1975). \emph{The Mythical Man-Month: Essays on Software
  Engineering}. Addison-Wesley.
\item
  Newman, S. (2015). \emph{Building Microservices: Designing
  Fine-Grained Systems}. O'Reilly Media.
\item
  Preston-Werner, T. (2013). Semantic Versioning 2.0.0.
  https://semver.org/
\end{enumerate}

\hypertarget{synthetic-biology-and-standardization-1}{%
\subsection{Synthetic Biology and
Standardization}\label{synthetic-biology-and-standardization-1}}

\begin{enumerate}
\def\labelenumi{\arabic{enumi}.}
\setcounter{enumi}{8}
\item
  Endy, D. (2005). Foundations for engineering biology. \emph{Nature},
  438(7067), 449-453.
\item
  Canton, B., Labno, A., \& Endy, D. (2008). Refinement and
  standardization of synthetic biological parts and devices.
  \emph{Nature Biotechnology}, 26(7), 787-793.
\item
  Galdzicki, M., Clancy, K. P., Oberortner, E., et al.~(2014). The
  Synthetic Biology Open Language (SBOL) provides a community standard
  for communicating designs in synthetic biology. \emph{Nature
  Biotechnology}, 32(6), 545-550.
\item
  Beal, J., Nguyen, T., Gorochowski, T. E., et al.~(2020). Communicating
  structure and function in synthetic biology diagrams. \emph{ACS
  Synthetic Biology}, 9(8), 2025-2040.
\item
  Knight, T. (2003). Idempotent vector design for standard assembly of
  BioBricks. \emph{MIT Artificial Intelligence Laboratory}.
\end{enumerate}

\hypertarget{organoids-and-tissue-engineering}{%
\subsection{Organoids and Tissue
Engineering}\label{organoids-and-tissue-engineering}}

\begin{enumerate}
\def\labelenumi{\arabic{enumi}.}
\setcounter{enumi}{13}
\item
  Lancaster, M. A., \& Knoblich, J. A. (2014). Organogenesis in a dish:
  Modeling development and disease using organoid technologies.
  \emph{Science}, 345(6194), 1247125.
\item
  Takebe, T., \& Wells, J. M. (2019). Organoids by design.
  \emph{Science}, 364(6444), 956-959.
\item
  Clevers, H. (2016). Modeling development and disease with organoids.
  \emph{Cell}, 165(7), 1586-1597.
\item
  Rossi, G., Manfrin, A., \& Lutolf, M. P. (2018). Progress and
  potential in organoid research. \emph{Nature Reviews Genetics},
  19(11), 671-687.
\end{enumerate}

\hypertarget{organ-on-chip}{%
\subsection{Organ-on-Chip}\label{organ-on-chip}}

\begin{enumerate}
\def\labelenumi{\arabic{enumi}.}
\setcounter{enumi}{17}
\item
  Bhatia, S. N., \& Ingber, D. E. (2014). Microfluidic organs-on-chips.
  \emph{Nature Biotechnology}, 32(8), 760-772.
\item
  Huh, D., Matthews, B. D., Mammoto, A., et al.~(2010). Reconstituting
  organ-level lung functions on a chip. \emph{Science}, 328(5986),
  1662-1668.
\item
  Ronaldson-Bouchard, K., \& Vunjak-Novakovic, G. (2018).
  Organs-on-a-chip: A fast track for engineered human tissues in drug
  development. \emph{Cell Stem Cell}, 22(3), 310-324.
\end{enumerate}

\hypertarget{systems-biology-modeling-1}{%
\subsection{Systems Biology Modeling}\label{systems-biology-modeling-1}}

\begin{enumerate}
\def\labelenumi{\arabic{enumi}.}
\setcounter{enumi}{20}
\item
  Hucka, M., Finney, A., Sauro, H. M., et al.~(2003). The systems
  biology markup language (SBML): A medium for representation and
  exchange of biochemical network models. \emph{Bioinformatics}, 19(4),
  524-531.
\item
  Lloyd, C. M., Halstead, M. D., \& Nielsen, P. F. (2004). CellML: Its
  future, present and past. \emph{Progress in Biophysics and Molecular
  Biology}, 85(2-3), 433-450.
\end{enumerate}

\hypertarget{biohybrid-systems-and-biorobotics}{%
\subsection{Biohybrid Systems and
Biorobotics}\label{biohybrid-systems-and-biorobotics}}

\begin{enumerate}
\def\labelenumi{\arabic{enumi}.}
\setcounter{enumi}{22}
\item
  Raman, R., \& Bashir, R. (2017). Biomimicry, biofabrication, and
  biohybrid systems: The emergence and evolution of biological design.
  \emph{Advanced Healthcare Materials}, 6(20), 1700496.
\item
  Ricotti, L., Trimmer, B., Feinberg, A. W., et al.~(2017). Biohybrid
  actuators for robotics: A review of devices actuated by living cells.
  \emph{Science Robotics}, 2(12), eaaq0495.
\item
  Park, S. J., Gazzola, M., Park, K. S., et al.~(2016). Phototactic
  guidance of a tissue-engineered soft-robotic ray. \emph{Science},
  353(6295), 158-162.
\item
  Cvetkovic, C., Raman, R., Chan, V., et al.~(2014). Three-dimensionally
  printed biological machines powered by skeletal muscle.
  \emph{Proceedings of the National Academy of Sciences}, 111(28),
  10125-10130.
\end{enumerate}

\hypertarget{brain-computer-interfaces}{%
\subsection{Brain-Computer Interfaces}\label{brain-computer-interfaces}}

\begin{enumerate}
\def\labelenumi{\arabic{enumi}.}
\setcounter{enumi}{26}
\item
  Musk, E., \& Neuralink. (2019). An integrated brain-machine interface
  platform with thousands of channels. \emph{Journal of Medical Internet
  Research}, 21(10), e16194.
\item
  Lebedev, M. A., \& Nicolelis, M. A. (2017). Brain-machine interfaces:
  From basic science to neuroprostheses and neurorehabilitation.
  \emph{Physiological Reviews}, 97(2), 767-837.
\end{enumerate}

\hypertarget{ethics-and-governance}{%
\subsection{Ethics and Governance}\label{ethics-and-governance}}

\begin{enumerate}
\def\labelenumi{\arabic{enumi}.}
\setcounter{enumi}{28}
\item
  Yuste, R., Goering, S., Arcas, B. A. Y., et al.~(2017). Four ethical
  priorities for neurotechnologies and AI. \emph{Nature}, 551(7679),
  159-163.
\item
  Ienca, M., \& Andorno, R. (2017). Towards new human rights in the age
  of neuroscience and neurotechnology. \emph{Life Sciences, Society and
  Policy}, 13(1), 5.
\item
  Hyun, I., Scharf-Deering, J. C., \& Lunshof, J. E. (2020). Ethical
  issues related to brain organoid research. \emph{Brain Research},
  1732, 146653.
\end{enumerate}

\hypertarget{general-references}{%
\subsection{General References}\label{general-references}}

\begin{enumerate}
\def\labelenumi{\arabic{enumi}.}
\setcounter{enumi}{31}
\item
  Simon, H. A. (1996). \emph{The Sciences of the Artificial} (3rd ed.).
  MIT Press.
\item
  Alexander, C., Ishikawa, S., \& Silverstein, M. (1977). \emph{A
  Pattern Language: Towns, Buildings, Construction}. Oxford University
  Press.
\item
  Kuhn, T. S. (1962). \emph{The Structure of Scientific Revolutions}.
  University of Chicago Press.
\end{enumerate}

\end{document}
